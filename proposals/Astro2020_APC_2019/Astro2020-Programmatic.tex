%%%%
% -- Proposed Work and Budget
% --     FOBOS Keck White Paper 2019
%%%%

% ========================================================================
% - leverage experience from Shane prime focus ADC; Harland design; Matt
%   Radovan may know; Dave Cowley probably best person to talk to

\subsection{Technology Drivers}
\label{sec:design}

FOBOS will provide key capabilities in the near-term thanks to
deployment at the existing Keck II telescope. It both carries a
relatively modest cost compared to other proposed large-scale
spectroscopic facilities (e.g., MSE, SpecTel) and helps lay the
groundwork for their realization. Thus, while FOBOS will prove to be
a valuable long-term investment for the W.~M.~Keck Observatory, it
can also provide for invaluable technological development leading to
efficiency and cost-cutting strategies for these larger facilities.

\subsubsection{Starbugs fiber positioners} Starbugs are a positioning
technology developed and deployed by Australian Astronomical Optics
(AAO), which has partnered with our team to generate a conceptual
design for use of Starbugs by FOBOS. The Starbugs positioning systems
is highly attractive because of its flexibility. This flexibility is
both in terms of configuring a given set of fiber, as well as the
prospect of exchanging different groups of Starbugs with different
payloads and/or those that feed different spectrographs (e.g., high
vs.\ low resolution). With such a flexible focal plane deployment,
FOBOS can serve as a platform for cost-cutting technology
development, which is not possible with fixed-format instruments like
PFS and DESI. Starbugs are currently being tested on-sky with the
TAIPAN instrument at the UK Schmidt Telescope and published results
on their performance are expected in summer 2019.

\subsubsection{Spectrograph Cost} \comment{to edit} Here we make the
case that FOBOS is a platform for figuring out how to build future
dedicated spectroscopic telescopes like SpecTel cheaper.

\subsubsection{Data Systems} A key to FOBOS's success with the
development of robust data-reduction and data-analysis pipelines,
building on the heritage of efforts within SDSS, DESI, and MaNGA. In
particular, the FOBOS data-analysis pipeline (DAP) will take
advantage of the fixed spectral format and common target classes to
provide high-level data products, including Doppler shifts,
emission-line strengths, and template continuum fits (cf., Westfall
et al.; SDSS-IV MaNGA DAP). Planning will include development of
user-friendly platforms built on the Keck Observatory Archive for
serving raw data, reduced spectra, and DAP science products.

% \comment{to edit/remove} We will develop the requirements and initial
% concept for the FOBOS data simulator following instrument
% forward-modeling techniques developed by DESI. The simulator will
% enable tests of potential FOBOS science cases (e.g., exposure time
% estimates, redshifting success). For the MSIP proposal, we will
% construct a detailed plan for evolving the data simulator into
% data-reduction and data-analysis pipelines.

\subsection{Current Status} FOBOS is currently in its conceptual design
phase, building from a down-selection process as one of the designs for
the Wide-Field Optical Spectrograph for TMT. Recently, FOBOS has been
awarded Phase-A funding from WMKO Observatory, receiving a full
design-phase endorsement of the Keck Science Steering Committee. These
funds are devoted building out the conceptual design in preparation for
future funding proposals, particularly the NSF MSIP and MsRI calls.

\subsection{Cost Estimates and Schedule}

Cost estimates for FOBOS reflect its current development phase.  Our
conceptual-design-phase estimates have higher fidelity for the near-term
phases of Preliminary Design and the beginning of Final Design, but have
less fidelity in the Construction and Commissioning phases.  Where
possible, costing efforts are based on quotes, and labor efforts are
based on experience with similar systems developed by the institution
responsible for a given sub-system.  A summary of the high-level costs
by project phase and the high-level schedule are included in Table
\ref{tab:cost}.  Our full project cost projections place FOBOS in the
category of a medium-scale ground-based program. % at \$37.5M.

Nearly all costs prior to Final Design 2 are allocated to design
efforts; however, a small amount is devoted to prototyping needed to
mitigate risk in key systems.  Our project execution plan divides Final
Design into two phases to allow for a gate for long-lead-time contracts
after a review in June of 2024, such as procurement and construction of
the ADC optics.  The Integration phase also overlaps with the
construction phase to allow for the facility system build-out at Keck
Observatory and to allow for a phased deployment of the multiplexed
focal-plane system.  A full project review is scheduled at the end of
each design phase.  A pre-ship review will be held during integration
prior to delivery of the first spectrograph.   Smaller sub-system
reviews will be held as required.

\begin{table}[h!]
\centering
\footnotesize
\caption{Nominal Schedule and Cost Estimates for FOBOS}
\label{tab:cost}
\vspace*{-10pt}
\begin{tabular}{l | l l r l }
\hline
Phase              &  Start  &     End &         Cost & Fidelity \\
                   &         &         &  (2019 kUSD) &  \\
\hline
\hline
Conceptual Design  & Q2 2018 & Q1 2021 &   730 & Resource-loaded schedule with progress tracking \\
Preliminary Design & Q2 2021 & Q2 2023 &  2580 & Resource-loaded schedule \\
Final Design One   & Q3 2023 & Q2 2024 &  1500 & High-level tasks with cost/effort estimates \\
Final Design Two   & Q3 2024 & Q2 2025 &  8000 & High-level tasks with cost/effort estimates \\
Construction       & Q3 2025 & Q1 2027 & 14000 & Block schedule with low-fidelity cost/effort estimates \\
Integration        & Q2 2026 & Q3 2027 & 10000 & Block Schedule with low-fidelity cost/effort estimates \\
Commissioning      & Q4 2027 & Q1 2028 &   700 & Block Schedule with low-fidelity cost/effort estimates \\
\hline
\end{tabular}
\end{table}









% \noindent \textbf{Operations:} Powered by Starbugs fiber positioners,
% FOBOS will enable fast ($<$2 minute), dynamic reallocation of fibers. We will
% develop an initial target allocation simulator to determine
% efficiencies for various science programs and explore options for
% program combination and optimization under different observing
% scenarios. This work requires planning interfaces with the fiber
% positioning control software and the Keck user.


% To efficiently
% determine the best options given a wide range of possible targets and
% desired observing outcomes, we will develop a conceptual design for
% MAISTRO,\footnote{MAISTRO: Modular Artificial Intelligence System for
% Target Reallocation and Observing.} an ``artificial intelligence''
% (AI) targeting system that will learn optimization strategies for
% assigning targets from a database of overlapping observing programs
% with pre-defined priorities. The AI package will aggregate data
% quality using a quick-look reduction package, science-driven
% performance metrics, {\it and real-time assessments of the observing
% conditions} to make dynamic targeting recommendations. For example,
% if conditions are slightly less than optimal, MAISTRO would
% reconfigure Starbugs to brighter objects in a field or implement a
% different program prioritization. MAISTRO will incorporate updated
% target lists and priorities from the active observer and could easily
% be over-ridden at any time. Fractions of the full FOBOS multiplex
% might also be reserved ``manual targeting'' as required by the
% program PI.

%   - maintains a database with observational progress on individual
%     targets in the survey and
%   - dynamically reallocates fibers based on real-time assessments of
%     the aggregate S/N of each target to meet the specific need of each
%     science case.

% This requires significant design and testing of a combined software
% package and hardware interface.  Specific considerations involve (1)
% fast and robust reduction procedures (cf. MaNGA DOS) that can assess
% the aggregate data and (2) a responsive database with a schema
% optimized for real-time decision making to select targets for
% (re)acquisition while accounting for collision limitations.  Provided
% enough design effort, this lends itself to a machine-learning
% application.

