%%%%
% -- Cosmology Science
% --     FOBOS Keck White Paper 2019
%%%%

\subsection{Enhancing Dark Energy Probes via Precision Cosmic Distances}
\label{sec:cosmology}

Enormous world-wide efforts --- culminating in LSST, Euclid, and
WFIRST --- are seeking highly precise measures of cosmic structure to
constrain the evolving dark-energy equation-of-state. These measures
utilize angular correlations of galaxy positions, their gravitational
lensing shear, and the cross-correlation between the two.
Unfortunately, photometric distances (via photometric redshifts, or
``photo-$z$s'') are significantly less accurate than spectroscopic
redshifts (spec-$z$s). An effective way to overcome photo-$z$ biases
is to calibrate the empirical relation between the colors measured in
a deep imaging survey and redshift by systematically sampling the
color space of galaxies with spectroscopy \citep{masters15}. The
Complete Calibration of the Color-Redshift Relation
\citep[C3R2][]{masters19} survey is designed to do just this for
Euclid. However, application of this approach LSST and WFIRST depths
will present new challenges \citep{hemmati18}. In these regimes,
FOBOS will play a critical role given its dramatically higher
multiplexing and unique sensitivity to the rest-UV features of
galaxies at z>1.5.

FOBOS offers spectroscopic validation of photo-$z$s that is therefore
critical to the success of {\it all} imaging surveys in this respect.
It would not only \emph{increase the dark energy figure-of-merit in
LSST by 40\%} \citep{newman15} but, importantly, provide vital
confidence in cosmological results. FOBOS is particularly powerful in
this application because it has no ``redshift desert'' thanks to its
unique ability to measure spectroscopic redshifts above $z > 1.5$ via
rest-frame UV features. This eliminates the need for expensive,
space-based\footnote{Ground-based near-IR spectroscopy is too
contaminated by sky-line emission to provide spec-$z$s at the
required level of completeness \citep{newman15}.} near-IR
spectroscopy.

\noindent\comment{LBG cosmology: Wilson, White?}

\noindent\comment{Kinematic Weak Lensing: Bundy, Huff, Schlegel, DiGiorgio?}