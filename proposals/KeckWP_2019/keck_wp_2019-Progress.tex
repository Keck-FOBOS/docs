%%%%
% -- Instrument Description
% --     FOBOS Keck White Paper 2019
%%%%


\section{Recent and Ongoing Progress}
\label{sec:progress}

FOBOS development over the past year has focused on (1) engaging the
scientific and instrumentalist communities in discussions about the
FOBOS design, particularly with regard to science cases and
instrument feasibility, (2) initial development of an instrument
simulator and exposure-time calculator, (3) advancing the
optomechanical designs of the focal-plane and ADC systems, (4)
consolidating development of Fiber-WFOS for TMT and repurposing much
of the design for FOBOS at Keck, (5) working with Keck instrument
scientists and engineers toward a practical integration plan, and (6)
developing a detailed project execution plan, schedule, and budget.
Many of these activities were enabled by funds provided in response
to our previous white-paper submission.

Visits to Keck-user institutes have been particularly helpful in
building interest around the instrument and refining its
specifications. As of this writing, Bundy, MacDonald, and/or Westfall
have visited UCR (also joined by Michael Cooper from UCI), UCLA, Keck
observatory, LBNL, and UCSB. We plan to continue these visits, hoping
to visit UCB, UCD, and CIT before the end of the year.

Three specific ways the instrument conceptual design has been
affected by these conversations are the emphasis on the flexibility
of the focal-plane sampling, the ``first-light'' multiplex capacity,
and sensitivity toward the UV. Although the capabilities of PFS are a
common comparison for FOBOS, many of the science goals now enumerated
in Section 1 require target densities and/or wavelength coverage that
PFS will not provide; in the overlapping wavelength range between the
two instruments, FOBOS's survey speed is \comment{x} times larger
than PFS's. Given both this and the SSC's own assessment of the need
for multiplex at Keck, our current design includes 1800 single-fiber
apertures over a 17' FOV (our goal remains to access the full 20'
FOV), yeilding a target density of $\sim$8 arcmin$^{-2}$
(approximately the target density of DEEP2), but able to meet a
maximum target density of $\sim$30 arcmin$^{-2}$ in patches of the
focal-plane given the flexibility of the Starbugs positioning system.
A strong message from Keck users, however, was that a large-format
monolithic IFU and many deployable IFUs were both highly desirable.
Again, this makes Starbugs attractive because other fiber-positioning
systems (e.g., robotic or Echidna-like zonal positioners) would make
swapping focal-plane formats much more difficult or untenable
\comment{check this}. Our initial concepts for these different
focal-plane formats provided a secondary motivation for FOBOS's large
pseudo-slit capacity: it can yield a monolithic IFU with a FOV that
is competitive with VLT/MUSE and can provide 15-100 smaller, freely
deployable IFUs. Finally, Keck remains unique in its throughput
toward the atmospheric limit. No other current or planned
\comment{check this again} spectrograph for 10m-class telescopes
provides sensitivity below $\sim$380nm, providing FOBOS with a unique
capability among the landscape of forthcoming instrumentation.

An identified risk of the front-end design of FOBOS is in the
coupling of the microlens fore-optics to the fiber and the coupling
of the microlens entrance aperture to the Keck II focal plane. It is
critical that both of these couplings minimize losses. Using funds
provided by a UCO mini-grant, we will address and begin to mitigate
these risks via prototype fiber builds that are tested in the fiber
labs at UCO and LBL; our plans include on-sky tests to verify these
lab results. Lab tests have already begun with imaging the near- and
far-field input and/or output beams of bare fibers. We have
optomechanical designs of the microlens-fiber coupling, and have
ordered hardware to be begin fabrication. We plan to propose for
experiments to be performed during engineering time at Keck by the
end of the year.

Finally, with the permission of the SSC, we proposed for
instrument-design funds from the NSF via its new Mid-scale Research
Infrastructure scheme. Although the proposal was ultimately
unsuccessful, preparation of the proposal led to a ground swell of
development in our project planning and science development, much of
which is included in this submission.
