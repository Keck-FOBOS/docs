%%%%
% -- Overview Material
% --     FOBOS Keck White Paper 2019
%%%%

\centerline{{\large\bf Executive Summary}}

% Soon, our community will be inundated with sources of interest from
% large-scale ground- and space-based surveys requiring spectroscopic
% follow-up for physical characterization. In fact, with the recent
% {\it Gaia} releases, the spectroscopic needs in Galactic Astronomy
% are already dire, a situation that will be acutely felt by
% extragalactic community once LSST, Euclid, and WFIRST begin
% delivering data. These sources will span the full range of scientific
% interests of our current Keck partners and Keck's current
% spectroscopic capabilities will not e sufficient to meet these needs.

The 2016 Keck Observatory Scientific Strategic Plan lists highly-multiplexed spectroscopy as a key desirable that would
position Keck to take advantage of the coming era of wide-field imaging facilities like LSST, Euclid, and WFIRST.  FOBOS addresses this priority via a fiber-based facility that optimizes depth over area, preserving Keck's historical strength in faint-object spectroscopy.  The result is a uniquely blue-sensitive, high-mutliplex instrument with order-of-magnitude greater sampling density than competitors like Subaru's Prime Focus Spectrograph (PFS).  In the LSST era, FOBOS will excel at building the deep, spectroscopic reference data sets needed to interpret vast imaging data.  At the same time, its flexible focal plane, including deployable integral field units (IFUs), enables an expansive range of scientific investigations from the diverse Keck community.

FOBOS will provide $R \sim 3500$ spectroscopy over an instantaneous bandpass of 310-1000 nm for as many as 1800
individual targets across a 17 arcminute diameter field.  The instrument is modular and composed of three major
components.  The focal plane system includes an atmospheric dispersion corrector (ADC) whose final optic traces the
Nasmyth focal surface.  This enables flexible target allocation by free-roaming Starbug positioners that ``walk'' on
this surface.  When configured in single-fiber mode, each Starbug carries a 150 $\mu$m core diameter fiber with
demagnifying fore-optics that samples a 0.9 arcsec diameter aperture on-sky.  If FOBOS is configured in IFU mode, a
different suite of Starbugs would be deployed on the focal plane.  These would carry IFU fiber bundles with coupled
lenslet arrays that provide finer spatial sampling.  A short fiber run (<10 m) through a stress-relief cabling system
minimizes throughput losses between the focal plane and an array of three temperature-controlled bench spectrographs
mounted on the Nasmyth platform adjacent to the focal plane system.

Its UV sensitivity, high multiplex and sampling density, and aperture flexibility make FOBOS compelling in science
areas that are traditional strengths of Keck Community.  These include Local Group archeology via resolved stellar
spectra at moderate spectral resolution in dwarf galaxies, M31, and the Milky Way halo.   FOBOS will be unparalleled in
providing detailed UV absorption line information and tomography for galaxies and the intergalactic medium at $z \sim
2$--4.  At $z \sim 1$, its depth and high sampling density will open new probes of galaxy environment while its IFU mode will enable kinematics studies of winds, the resolved gas-phase properties and dynamics of star-forming galaxies, and internal structure of stellar populations.

High-multiplex and deep spectroscopic followup of LSST and other
panoramic deep-imaging surveys is a widely recognized necessity.
Reports in 2015 and 2016 by the National Research Council and
National Optical Astronomical Observatory specifically recommended
investment in new spectroscopic facilities to meet these needs
because none currently exist or are planned for U.S.\ observatories.
We seek seed funding from WMKO to continue conceptual design of
FOBOS, a powerful new spectrograph for the Keck II telescope that increases its current survey speed by a factor of X.

Led by NSF's Large Synoptic Survey Telescope\footnote{
%
LSST will be begin science operations in 2023.}
%
(LSST), astronomy is entering a new era of unprecedented deep-imaging data sets that will survey huge volumes of the
Universe.  From the emergence of the earliest galaxies from a ``primordial soup'' of gas and dust, to the peak of
cosmic star formation and the current era of accelerated expansion, these surveys will provide unprecedented statistics
at key epochs of cosmic history.


Even so, gaining physical insight from panoramic imaging surveys will require intensive spectroscopic follow-up.  The
power of combining photometry and dedicated spectroscopy is widely appreciated and perhaps best illustrated by the
success of the Sloan Digital Sky Survey (SDSS) which used this combination to record the properties of over 1 million
galaxies, mapping the present-day universe and making SDSS one of the most highly cited surveys in the history of
astronomy.

LSST's all-sky images will be 1,000 times deeper and detect far more
distant galaxies than SDSS, but \textbf{no current U.S. facility is
capable of obtaining spectroscopic follow-up of LSST galaxies} at a level
required to capitalize on the \$1B the U.S.\ has invested in that
project.  In fact, an SDSS-like spectroscopic study of 1 million
galaxies at LSST depth would require 300 years of observing on the
largest telescopes with current instrumentation!  

Solving this problem requires not only more powerful spectroscopic facilities, but new ways to take better advantage of
what will be necessarily more limited spectroscopy compared to the vast imaging surveys of the LSST era.  The more
promising path is encapsulated in one of NSF's ``10 Big Ideas,'' \emph{Harnessing the Data Revolution}: we can maximize
the information content of LSST and other imaging facilities via machine learning from optimally designed spectroscopic
training sets.

This proposal presents a coordinated framework with three critical
components necessary for success in this endeavor: 1) Using simulated
spectroscopic$+$imaging data to define the training sets required to
address ambitious data-science challenges in Cosmology, Galaxy
Formation, and Local Group Archaeology in the LSST era; 2) Preliminary
design of FOBOS,\footnote{FOBOS: Fiber-Optic Broadband Optical
Spectrograph} a state-of-the-art spectroscopic facility for
WMKO,\footnote{WMKO: W.~M.\ Keck Observatory operates the two twin 10m
Keck Telescopes on Maunakea, Hawaii.} optimized for providing critical
training data using one of the world's largest telescopes; 3)
Preliminary design of the coordinated FOBOS observations required as
well as the systems needed to serve training data publicly.  This {\bf
MSRI-1 Design} proposal lays out the path for maximizing panoramic
imaging from LSST, WFIRST,\footnote{WFIRST is NASA's space-based
Wide-Field Infrared Survey Telescope, expected to launch in the mid
2020's.} Euclid,\footnote{Euclid is led by the European Space Agency
with significant NASA involvement and will launch in 2021. Its primary
mission is a 15,000 deg$^2$ imaging survey in optical and near-IR
wavebands.} and other facilities with spectroscopic follow-up
unparalleled in depth and sampling density.  Through a subsequent MSRI
proposal we will deliver on our goals with an instrument deployment in
2026, an array of spectroscopic programs, and associated public-ready
training data.



Led by NSF's Large Synoptic Survey Telescope\footnote{
%
LSST will be begin science operations in 2023.}
%
(LSST), astronomy is entering a new era of unprecedented deep-imaging data sets that will survey huge volumes of the
Universe.  From the emergence of the earliest galaxies from a ``primordial soup'' of gas and dust, to the peak of
cosmic star formation and the current era of accelerated expansion, these surveys will provide unprecedented statistics
at key epochs of cosmic history.

% Meanwhile, the rate of cosmic expansion was beginning to accelerate,
% as the Universe became increasingly dominated by ``Dark Energy,''
% whose origin remains the single greatest mystery in astronomy and
% cosmology today.

% Since Edwin Hubble's observations over 100 years ago,

Even so, gaining physical insight from panoramic imaging surveys will require intensive spectroscopic follow-up.  The
power of combining photometry and dedicated spectroscopy is widely appreciated and perhaps best illustrated by the
success of the Sloan Digital Sky Survey (SDSS) which used this combination to record the properties of over 1 million
galaxies, mapping the present-day universe and making SDSS one of the most highly cited surveys in the history of
astronomy.

% Because a quality spectrum requires far more observing time per source
% than an image, SDSS pioneered ``high multiplex'' spectrographs,
% capable of \emph{simultaneous} spectroscopy of hundreds of objects.

LSST's all-sky images will be 1,000 times deeper and detect far more
distant galaxies than SDSS, but \textbf{no current U.S. facility is
capable of obtaining spectroscopic follow-up of LSST galaxies} at a level
required to capitalize on the \$1B the U.S.\ has invested in that
project.  In fact, an SDSS-like spectroscopic study of 1 million
galaxies at LSST depth would require 300 years of observing on the
largest telescopes with current instrumentation!  

Solving this problem requires not only more powerful spectroscopic facilities, but new ways to take better advantage of
what will be necessarily more limited spectroscopy compared to the vast imaging surveys of the LSST era.  The more
promising path is encapsulated in one of NSF's ``10 Big Ideas,'' \emph{Harnessing the Data Revolution}: we can maximize
the information content of LSST and other imaging facilities via machine learning from optimally designed spectroscopic
training sets.

This proposal presents a coordinated framework with three critical
components necessary for success in this endeavor: 1) Using simulated
spectroscopic$+$imaging data to define the training sets required to
address ambitious data-science challenges in Cosmology, Galaxy
Formation, and Local Group Archaeology in the LSST era; 2) Preliminary
design of FOBOS,\footnote{FOBOS: Fiber-Optic Broadband Optical
Spectrograph} a state-of-the-art spectroscopic facility for
WMKO,\footnote{WMKO: W.~M.\ Keck Observatory operates the two twin 10m
Keck Telescopes on Maunakea, Hawaii.} optimized for providing critical
training data using one of the world's largest telescopes; 3)
Preliminary design of the coordinated FOBOS observations required as
well as the systems needed to serve training data publicly.  This {\bf
MSRI-1 Design} proposal lays out the path for maximizing panoramic
imaging from LSST, WFIRST,\footnote{WFIRST is NASA's space-based
Wide-Field Infrared Survey Telescope, expected to launch in the mid
2020's.} Euclid,\footnote{Euclid is led by the European Space Agency
with significant NASA involvement and will launch in 2021. Its primary
mission is a 15,000 deg$^2$ imaging survey in optical and near-IR
wavebands.} and other facilities with spectroscopic follow-up
unparalleled in depth and sampling density.  Through a subsequent MSRI
proposal we will deliver on our goals with an instrument deployment in
2026, an array of spectroscopic programs, and associated public-ready
training data.

The need for spectroscopic follow-up in the LSST era was made clear in
the National Research Council's 2015 report, ``Optimizing the U.S.
Ground-Based Optical and Infrared Astronomy System'' \citep{NAP21722}:
%
\noindent\begin{center}\mbox{\parbox{0.95\linewidth}{
%
The National Science Foundation should support the development of a
wide-field, highly multiplexed spectroscopic capability on a medium- or
large-aperture telescope in the Southern Hemisphere to enable a wide
variety of science, including follow-up spectroscopy of Large Synoptic
Survey Telescope targets. Examples of enabled science are studies of
cosmology, galaxy evolution, quasars, and the Milky Way.
%
}}\end{center}

Workshops organized by the National Optical Astronomy Observatory (NOAO)
in 2013 and 2016, the latter at the NSF's request, reported specific
spectroscopic needs for LSST follow-up in all science areas.  In
particular, the 2016 report notes that a critical resource in need of
prompt development is to ``Develop or obtain access to a highly
multiplexed, wide-field optical multi-object spectroscopic capability on
an 8m-class telescope.''  Based on these recommendations, we propose the
FOBOS instrument coupled with a suite of data-driven tools to address
the spectroscopic requirements of LSST and other photometric surveys at
a final cost 20 times less than a new Southern Hemisphere facility.
Located in Hawaii, FOBOS can access more than 70\% of the LSST
footprint, more than adequate for building powerful
spectroscopic training sets.  Compared to the Prime Focus Spectrograph
(PFS) on Japan's Subaru Telescope, FOBOS would be 1.7$\times$ faster,
provide unique UV sensitivity (0.31--1 $\mu$m compared to
0.38--1.25 $\mu$m with PFS), and offer higher-density, more flexible
target sampling over ``deep-drilling'' fields.  Unlike PFS, FOBOS would be operated
on a U.S.\ telescope with dedicated U.S.\ access and a commitment to
supporting U.S.-led imaging facilities.  FOBOS is also complementary to
future telescopes and instruments that would be optimized to cover wider areas
(several deg$^2$ per pointing) at shallower depths.

%\comment{mention FOBOS can do PI-led science too}

The need for deep spectroscopic follow-up is particularly acute for the major cosmological probes to be carried out by
LSST, Euclid, and WFIRST, which all rely on ``photometric redshifts:'' measures of galaxy redshift, $z$
--- a direct proxy of distance and look-back time---based on imaging alone. \citet{newman15} summarize the case for a
    significant spectroscopic campaign to calibrate and train LSST photometric redshifts in order to improve cosmological constraints.  They describe a redshift survey that,
    if carried out with FOBOS, would increase LSST's Dark Energy figure-of-merit by 40\% at a cost of less than 5\% of
    the LSST budget.  The urgent case for spectroscopic redshift training has been the subject of numerous publications
    \citep[e.g.,][]{laureijs11, masters15, hemmati18}.

Meanwhile, the astronomy community recognizes that the coming ``Big
Data'' era, culminating in LSST, necessitates ``\textbf{harnessing the
data revolution}.''  Widespread community interest in advanced
data-science techniques continues to grow amidst calls for educational
programs, conference series, and research funding to support the growth
of a new field, ``Astroinformatics,'' which exploits the interface
between astrophysics and statistics \citep{borne09}.  Astronomy's
largest organizations, including the American Astronomical Society and
the International Astronomical Union, have supported active working
groups on astroinformatics and astrostatistics since 2015.  LSST itself
has supported the Informatics and Statistics Science Collaboration and
partnered with NSF on the Data Science Fellowship Program to train
astronomy graduate students in data-science techniques.  Our proposal
builds on and contributes to these ongoing efforts.
