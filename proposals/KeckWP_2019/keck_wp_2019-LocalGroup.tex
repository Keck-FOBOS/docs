%%%%
% -- Local Group Science Cases
% --     FOBOS Keck White Paper 2019
%%%%

\subsection{Unraveling the Formation History of our Local Group of Galaxies}
\label{sec:localgroup}

Our Local Group of galaxies---the Milky Way (MW), the Magellanic Clouds,
Andromeda (M31) and Triangulum (M33) Galaxies, and a multitude of
satellite galaxies---allows us to study one realization of the galaxy
formation process in superb detail.  In the next decade, LSST and WFIRST
will increase the census of stellar streams and halo substructure in
these galaxies by a hundredfold.  Follow-up \emph{stellar} spectroscopy
will constrain stream orbits and the total mass they enclose
\citep{2017ApJ...836..234S} as well as the associated age and chemical
composition (see below).
%\citep{2019MNRAS.484.3425M}.
As theoretical modeling also advances, these new data promise exciting
insights on the formation history of the Local Group.

Radial velocity studies of stars in the MW halo or the M31 disk require
observations of up to 10 hours on large telescopes
\citep[e.g.,][]{2018arXiv180904082C}.  This again motivates
machine-learning algorithms to extract physical quantities from both
multi-band imaging and lower quality spectra (low resolution and S/N)
using relatively small, yet high-S/N, training sets.  For example,
\citet{2015ApJ...808...16N} have developed {\it The Cannon}, a
supervised learning approach that uses spectra with known stellar
parameters to label spectra where those parameters are unknown
(Fig.~\ref{fig:Cannon}).  Additionally, \citet{2018arXiv180401530T} have
developed {\it The Payne} which can infer 16 stellar-abundance labels
from low-resolution spectra using a neural network and theoretical
stellar spectra.  Finally, \citet{2018arXiv180803278T} have combined
Kepler-based astroseismology measurements with APOGEE spectra to
determine stellar age to $\sim$25\% precision using a neural network.
Our proposed effort builds on new lines of inquiry based on these
successes.

\begin{enumerate}[rightmargin=0.2cm,leftmargin=0.2cm]

\chal{stellar} 
%
\item[] {\textsf {\large  Data-Science Challenge \ref{stellar}: A nested network of stellar parameter training
samples for resolved Milky Way and Local Group studies.}}  As in our other challenges, the key driver here is the
ability to extract maximum information from photometry, in this case stellar parameters.  Our goal is to reach
magnitudes significantly fainter than the detection limit of current and upcoming spectroscopic surveys of the Milky
Way including Gaia, APOGEE,\footnote{APOGEE, the Apache Point Observatory Galaxy Evolution Experiment has observed in
both SDSS-III and SDSS-IV.} the SDSS-V Milky Way Mapper, planned programs with 4MOST\footnote{4MOST: 4-meter
Multi-object Spectroscopic Telescope.} and the Dark Energy Spectroscopic Instrument (DESI) Milky Way Survey, among
others. Inferring
stellar parameters beyond V$\sim$18 will open up studies of the Milk Way's outer halo, the halo of M31, and stellar
populations in local dwarf galaxies.

The immediate challenge is to design an optimized, nested set of training samples that connect data from the surveys
above.  This nested set will span high-S/N to low-S/N and high spectral resolution to low spectral resolution for
sufficiently large, overlapping stellar samples.  Subsets will have astroseismology from TESS\footnote{TESS is NASA's
Transiting Exoplanet Survey Satellite.} and PLATO.\footnote{PLATO is ESA's PLAnetary Transits and Oscillations
mission.}  Using simulated spectra with known input parameters, we will test methods for ``label transfer'' from
information-rich spectra to information-poor spectra as we work down to fainter magnitudes, landing eventually at
multi-band photometry alone. Within this nested set, low-resolution FOBOS data will fill in gaps at both high-S/N,
where we will be training FOBOS data on higher resolution spectroscopy, as well as lower-S/N where we will be training
photometry on FOBOS spectroscopy.  The success of this multi-layered label transfer depends not only on the size of the
training sets we can access or observe, but on how representative they are.  Label transfer to WFIRST imaging of the
M31 halo, or Local Group dwarfs in either hemisphere, is a particular concern.  We will test it by evaluating label
recovery on simulated stellar spectra with cosmologically-informed formation histories for M31 and dwarf galaxies,
suitably differentiated from the Milky Way stars that anchor the training network.



% \chal{mwhalo} 
% %
% \item[] {\textsf {\large  Data-Science Challenge \ref{mwhalo}: The
% chemical evolution and assembly history of the MW stellar halo.}}  Using current MW halo models, we will simulate FOBOS
% stellar spectroscopy of main-sequence turn-off
% and red-giant stars in these substructures within the MW that also
% leverages existing data from, e.g., APOGEE and H3.  We will build
% data-driven models based on these data to measure stellar parameters
% (temperature, surface gravity, metallicity, and alpha-element abundance)
% for all halo stars with LSST+2MASS+WISE+WFIRST multi-band photometry,
% allowing us to reconstruct the star-formation history of each disrupted
% satellite. These will be combined with dynamical data and compared with
% cosmological simulations to build a generative model for the assembly
% history of the MW stellar halo.

% \chal{m31} 
% %
% \item[] {\textsf {\large Data-Science Challenge \ref{m31}: The
% differential chemical evolution of M31 and MW.}}  A natural extension of
% Data-Science Challenge \ref{mwhalo} is to perform the same analysis for the
% halo of M31.  However, we cannot expect to obtain high-quality spectra
% of individual main-sequence stars at the distance of M31 with FOBOS.
% Moreover, training a chemical evolution model using spectra of Milky Way
% stars may lead to systematic errors:  The Milky Way and Andromeda have
% distinct evolutionary histories \citep[e.g.][]{2005MNRAS.356.1071R},
% despite being relatively similar in many other respects.  We will
% therefore obtain deep observations of giant stars in the M31 halo to
% drive a machine-learning algorithm that combines a model of the MW halo
% with results from cosmological hydrodynamical simulations to constrain
% the differential history of the MW and M31 stellar halos.

% \chal{gaia} 
% %
% \item[] {\textsf {\large Data-Science Challenge \ref{gaia}: Stellar
% parameter determinations for a billion stellar spectra.}} While
% providing on-sky motions and photometry for 1.7 billion stars in the MW,
% fewer than 10\%, 0.3\%, and 0.1\% of stars will have a full complement
% of astrometrics and kinematics, basic stellar parameters, and chemical
% abundances, respectively.  Moreover, Gaia distance errors increase
% quadratically with distance.  To realize Gaia's full potential, we will
% design FOBOS training sets that, when combined with high-resolution
% datasets from, e.g., APOGEE, WEAVE, will allow us to build data-driven
% models of the absolute magnitude (yielding distance modulus),
% temperature, surface-gravity, and stellar abundance for {\it all} stars
% in the Gaia dataset.  These data will allow us to isolate coeval
% populations in the Galactic disk that can be combined with very
% high-resolution simulations of the Milky Way to provide a detailed
% evolutionary history of our Galactic home.

\end{enumerate}
