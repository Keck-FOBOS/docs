%%%%
% -- Proposed Work and Budget
% --     FOBOS Keck White Paper 2019
%%%%

\section{Proposed Work}
\label{sec:design}

The immediate need for FOBOS development is design funding that leads
to compelling proposals for significant investiment from the NSF
and/or a private foundation. We outline our funding strategy
\comment{somewhere}, and we make the following two requests of WMKO
and the Keck SSC: \textbf{
\begin{enumerate}
\item We request approval to pursue our funding path at the level
needed to fully fund FOBOS through design, construction, integration,
and commissioning.
\item We request seed funding of \comment{xx} over the next two years
(FY 2020 and 2021) needed to advance the instrument design to the
level needed to prepare competitive proposals for upcoming NSF MSIP
and MsRI solicitations.
\end{enumerate}
}

\subsection{Instrument Design Work} We list below the primary
instrument-design activities that we have identified as top
priorities for FOBOS development in the lead up to larger funding
requests. The schedule and budget for these activities are outlined
in Section \comment{additional pages}.

\noindent \textbf{Atmospheric Dispersion Compensator (ADC):} The
optomechanical design, tolerancing, lens cell design, motion systems,
and software-control design of the ADC will be completed.  

\noindent \textbf{Focal Plane System:} Mechanical design, including
flexure analysis and the selection of drive mechanisms and potential
vendors will be completed. This system also defines one of the
interfaces to the Keck II Telescope and must comply with WMKO space
envelopes, servicing needs, and other requirements. The focal plane
system also inputs the guide cameras.

\noindent \textbf{Starbugs fiber positioners:} Starbugs are a
positioning technology developed and deployed by the Australian
Astronomical Observatory (AAO), which has partnered with our team to
generate a conceptual design for use of Starbugs by FOBOS.  Design
requirements for Starbugs in FOBOS are more relaxed than the currently
on-sky TAIPAN instrument thanks to the larger physical plate scale at
Keck.  

\noindent \textbf{Fiber System:} We will complete the optical design and
processing plan for affixing forward optics lenses to each fiber head.  A
micro-lens array solution will be developed for a central,
fixed-position 4.5-arcsec diameter IFU for fast source acquisition. This
work package also inputs the stress-relief cable system and fiber
termination hardware and processing.

\noindent \textbf{Spectrographs.} The optical systems and components
(slit, collimator, dichroics, gratings, and camera), an analysis of
acceptable tolerances and performance, their mechanical supports,
software controls, and the overall enclosure will all be advanced
through Preliminary Design.  Detectors, cryostats, read-out electronics
and systems for thermal management will be designed.

% Put the calibration system back in?
% \noindent \textbf{Calibration System.} This package inputs design of an interior dome screen and projection system for injecting calibration sources with sufficient spatial uniformity and stability into the instrument.  We will work with the Observatory to develop an integration and controls plan.  No such calibration system currently exists at Keck.

% \noindent \textbf{Auxiliary Systems.} Design of auxiliary systems inputs Nasmyth platform interfaces, utilities access, fiber routing and support, thermal control and vibration control systems.

\subsection{MAISTRO: Target Allocation with Artificial Intelligence}
\label{sec:targeting}

\comment{edit/remove this?} Powered by Starbugs fiber positioners, FOBOS
will enable fast, dynamic reallocation of fibers. To efficiently
determine the best options given a wide range of possible targets and
desired observing outcomes, we will develop a preliminary design for
MAISTRO,\footnote{MAISTRO: Modular Artificial Intelligence System for
Target Reallocation and Observing.} an ``artificial intelligence''
(AI) targeting system that will learn optimization strategies for
assigning targets from a database of overlapping observing programs
with pre-defined priorities. The AI package will aggregate data
quality using a quick-look reduction package, science-driven
performance metrics, {\it and real-time assessments of the observing
conditions} to make dynamic targeting recommendations. For example,
if conditions are slightly less than optimal, MAISTRO would
reconfigure Starbugs to brighter objects in a field or implement a
different program prioritization. MAISTRO will incorporate updated
target lists and priorities from the active observer and could easily
be over-ridden at any time. Fractions of the full FOBOS multiplex
might also be reserved ``manual targeting'' as required by the
program PI.

%   - maintains a database with observational progress on individual
%     targets in the survey and
%   - dynamically reallocates fibers based on real-time assessments of
%     the aggregate S/N of each target to meet the specific need of each
%     science case.

% This requires significant design and testing of a combined software
% package and hardware interface.  Specific considerations involve (1)
% fast and robust reduction procedures (cf. MaNGA DOS) that can assess
% the aggregate data and (2) a responsive database with a schema
% optimized for real-time decision making to select targets for
% (re)acquisition while accounting for collision limitations.  Provided
% enough design effort, this lends itself to a machine-learning
% application.

\subsection{Automated Data Products}
\label{sec:DAP}

\comment{edit/remove this?} While the FOBOS data simulator is required for
our data-science challenges, it also forms the basis of a delivered
data reduction pipeline (DRP) for this instrument. This software will
provide both the quick reduction assessments needed for dynamic
targeting, as well as full reductions for scientific analysis. In the
proposal period, we will also develop a preliminary design for a data
analysis pipeline (DAP). Unique among Keck instruments, the FOBOS DAP
will take advantage of the fixed spectral format and common target
classes to provide high-level data products, including Doppler shift,
emission-line strengths, and template continuum fits (cf., Westfall
et al.; SDSS-IV MaNGA DAP). The DAP will also produce results from
relevant machine-learning applications (e.g., redshifts at low-S/N).

Raw data, reduced spectra, and high-level DAP science products will
be publicly delivered via user-friendly platforms built on the Keck
Observatory Archive. After associated proprietary periods, data will
be served for {\it all} FOBOS observations, creating a rich legacy
data set for the astronomical community. Both program PIs and the
larger community will be encouraged to develop the DRP and DAP to
meet the needs of specific science applications. These software
packages will be open source and publicly served (e.g., using
GitHub).
