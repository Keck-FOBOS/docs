%%%%
% -- Galaxies Science Cases
% --     FOBOS Keck White Paper 2019
%%%%

\subsection{Globular cluster populations of galaxies}

Presently, it is very expensive to conduct systematic spectroscopic
studies of the various galaxy types in rich galaxy clusters, like
Coma, due to their angular spread on the sky. With FOBOS's flexible
fiber-positioning system and 17-arcminute FOV, it will be possible to
simultaneously (and efficiently) build up an unprecedented library of
spectroscopic redshifts and stellar-population parameters of galaxies
in clusters towards intermediate redshift. Follow-up FOBOS
observations using its deployable mini-IFUs will allow us to
simultaneously obtain resolved spectroscopy for 10s of these cluster
galaxies, enabling us to associate internal structures/properties of
the galaxies with their host cluster.

\subsection{The properties and coevolution of galaxies and galaxy clusters}

Similarly, probing the outskirts of nearby galaxies using globular
clusters (GCs) and/or planetary nebulae (PNe) is tedious and
inefficient with current facilities due to the sparse azimuthal
distribution of these chemodynamical tracers at large radii. FOBOS
will be a game changer in the study of the dynamical mass
distributions of nearby galaxies with $\mathcal{M_\ast/M_\odot}
\lesssim 10^{11}$, which typically host $\lesssim1500$ GCs
\citep{2013ApJ...772...82H}. This is because FOBOS makes it possible
to acquire spectra for nearly all GCs located to $\sim$50 kpc from
their host galaxy in a single night. These data will allow us to map
the chemodynamics of massive galaxy halos and infer orbital families
as a function of stellar-population properties. Additionally, we can
more quickly build statistically relevant models for GC formation in
the context of the larger galaxy population.

\subsection{Resolved spectroscopy of galaxies toward $z$$\sim$1}

% By the present epoch, the majority of galaxies have settled into 

% The slow, steady aging and dynamical settling of evolution caused largely by local phenomena will be the
% dominant evoprimary method that will drive galaxy evolution in the future.  However, 

% These prospects become richer when/if a facility-level Ground-Layer
% Adaptive Optics (GLAO) is installed at Keck.

\noindent\comment{Westfall, Bundy, Max -- Resolved spectroscopy}

\subsection{The proto-galaxy ecosystem at $z$$\sim$2}
\label{sec:z2galaxies}

At roughly three billion years after the Big Bang, proto-galaxies in
the Universe transitioned from turbulent, gas-rich systems to the
more ordered, star-dominated structures of the current epoch. This
transition also marked the peak of cosmic star formation and galaxy
assembly. A deep understanding of this key epoch requires a
statistical characterization of the entire galaxy ``ecosystem,''
including not only the galaxies themselves but their gas-filled
environments, allowing us to to build a comprehensive picture of the
physical processes that fuel proto-galaxy growth, shape their
internal structure, and influence their environment.

Statistical benchmarks, like the Sloan Digital Sky Survey
\citep[SDSS][]{2000AJ....120.1579Y} at $z$$\sim$0 and DEEP2
\citep{2003SPIE.4834..161D, 2013ApJS..208....5N} at $z$$\sim$1, have
revolutionized our understanding of galaxy evolution during and
between these epochs. However, a comparable study at $z$$\sim$2 has
remained out of reach.  Although this will likely be true once it is deployed,
FOBOS provides more than an order of
magnitude increase in survey speed compared to DEIMOS, allowing one
to build up samples of $\gtrsim$10$^4$ with a modest investment. More
exciting, however, is the prospect of combining FOBOS spectroscopy
with recently developed machine-learning techniques to build a
1M-object survey a these redshifts.  With improved photo-$z$s and strong priors on
spectral types, our challenge is to push machine-learning
techniques to deliver {\it spectroscopic} redshifts (with $\lesssim$300 km/s
s accuracy) at the lowest signal-to-noise possible. Reductions
by factors of 4--5 in exposure time would enable FOBOS to complete a
1M galaxy environment survey at $z=1$--$2$ in just 20-30 nights.
%, 2011AJ....142...72E, 2017AJ....154...28B}

\noindent\comment{Cooper, Bundy}

\subsection{Ly$\alpha$ morphology and kinematics of lensed, magnified
galaxies at $z$$\sim$2--3}

\noindent\comment{Siana}

\subsection{The budget of ionizing photons at $z$$\gtrsim$2.5}

\noindent\comment{Shapley, Siana}

\subsection{The volume density and chemistry of gas in between galaxies}

\noindent\comment{Prochaska, Burchett}

\subsection{Tomography of the IGM}

\noindent\comment{Lee, Hennawi}

\subsection{Quasar Light Echos} 

\citep{2018arXiv181005156S}

\comment{Hennawi, Schmidt}


% From George:
% - fill out case for probing both galaxies and their “gas-filled
%   environments”
%    - make it more explicit that getting large numbers of redshifts
%      would make it possible to trace out large-scale structure in
%      detail
%    - enables studies of galaxy properties as a function of environment
%
% - also mention targeting galaxies along QSO lines of sight
%    - much higher target density than with LRIS, DEIMOS over larger FOV.
%
% - Worth discussing Lyman-alpha or metal-line tomography?  
%
% - More quantitative comparisons with existing data sets?
%    - What key science questions can FOBOS address that many years of
%      LRIS and DEIMOS observations have not been able to?  Surely some
%      level of the spectral tagging and photo-z training can be done
%      (and surely is being done) with existing data.  Is FOBOS going to
%      be a huge leap, or will it mainly be cleaning up neglected corners
%      of parameter space?
%
% - More excited to hear about how the FOBOS spectra will be used for
%   science directly, instead of support for LSST

%-----------------------------------------------------------------------
