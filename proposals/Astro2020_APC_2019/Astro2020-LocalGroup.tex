%%%%
% -- Local Group Science Cases
% --     FOBOS Keck White Paper 2019
%%%%

\subsubsection{Resolved studies of Local Group galaxies}
%Unraveling the Formation History of our Local Group of Galaxies}
\label{sec:localgroup}

Studies of individual stars in the Milky Way (MW), Magellanic Clouds,
Andromeda (M31), Triangulum galaxy (M33), and numerous dwarf
satellites provide an exquisitely detailed look at specific examples
of galaxy assembly and evolution. Panoramic imaging data (e.g., LSST
and WFIRST) will increase the census of stellar streams and other
substructure in these local systems by a hundredfold. To constrain
stream orbits and infer their enclosed mass
\citep{2017ApJ...836..234S} as well as diagnose their age and
chemical composition, FOBOS's spectroscopic depth and high multiplex
is critical. Complementing faint {\it Gaia} targets, FOBOS will
enable machine-learning for photometric stellar parameter recovery
\citep[see Section \ref{sec:datascience};][]{2015ApJ...808...16N,
2018arXiv180401530T, 2018arXiv180803278T} in service of direct probes
of radial migration, disk heating, and other assembly processes in
the MW and M31.


% FOBOS spectroscopy of
% these structures will, e.g., constrain stream orbits and the total
% mass they enclose . These data will also
% provide age and chemical composition measurements either directly or
% via targeted machine-learning applications . FOBOS is well-suited to these observations by
% dramatically improving the survey speed over DEIMOS (by an order of
% mangitude or more) and will remain sensitive enough to push toward
% the main-sequence turn-off of MW substructures. Compared to the MW,
% studies of M31's disk and halo are more suited for dedicated FOBOS
% programs. Indeed, the dramatic improvement in survey speed will,
% e.g., yield a direct measurements of secular processes active in
% M31's disk, such as radial migration and vertical disk heating
% \citep{2013ApJ...779..103D, 2015ApJ...803...24D,
% 2019ApJ...871...11Q}. Even so, {\it Gaia} targets, particularly at
% larger distances and higher target densities in the bulge, bar and
% inner disk, will prove uniquely suitable for FOBOS spectroscopy.

%\comment{Guhathakurta, Rockosi, Weisz}

% \chal{mwhalo} 
% %
% \item[] {\textsf {\large  Data-Science Challenge \ref{mwhalo}: The
% chemical evolution and assembly history of the MW stellar halo.}}  Using current MW halo models, we will simulate FOBOS
% stellar spectroscopy of main-sequence turn-off
% and red-giant stars in these substructures within the MW that also
% leverages existing data from, e.g., APOGEE and H3.  We will build
% data-driven models based on these data to measure stellar parameters
% (temperature, surface gravity, metallicity, and alpha-element abundance)
% for all halo stars with LSST+2MASS+WISE+WFIRST multi-band photometry,
% allowing us to reconstruct the star-formation history of each disrupted
% satellite. These will be combined with dynamical data and compared with
% cosmological simulations to build a generative model for the assembly
% history of the MW stellar halo.

% \chal{m31} 
% %
% \item[] {\textsf {\large Data-Science Challenge \ref{m31}: The
% differential chemical evolution of M31 and MW.}}  A natural extension of
% Data-Science Challenge \ref{mwhalo} is to perform the same analysis for the
% halo of M31.  However, we cannot expect to obtain high-quality spectra
% of individual main-sequence stars at the distance of M31 with FOBOS.
% Moreover, training a chemical evolution model using spectra of Milky Way
% stars may lead to systematic errors:  The Milky Way and Andromeda have
% distinct evolutionary histories \citep[e.g.][]{2005MNRAS.356.1071R},
% despite being relatively similar in many other respects.  We will
% therefore obtain deep observations of giant stars in the M31 halo to
% drive a machine-learning algorithm that combines a model of the MW halo
% with results from cosmological hydrodynamical simulations to constrain
% the differential history of the MW and M31 stellar halos.

% \chal{gaia} 
% %
% \item[] {\textsf {\large Data-Science Challenge \ref{gaia}: Stellar
% parameter determinations for a billion stellar spectra.}} While
% providing on-sky motions and photometry for 1.7 billion stars in the MW,
% fewer than 10\%, 0.3\%, and 0.1\% of stars will have a full complement
% of astrometrics and kinematics, basic stellar parameters, and chemical
% abundances, respectively.  Moreover, Gaia distance errors increase
% quadratically with distance.  To realize Gaia's full potential, we will
% design FOBOS training sets that, when combined with high-resolution
% datasets from, e.g., APOGEE, WEAVE, will allow us to build data-driven
% models of the absolute magnitude (yielding distance modulus),
% temperature, surface-gravity, and stellar abundance for {\it all} stars
% in the Gaia dataset.  These data will allow us to isolate coeval
% populations in the Galactic disk that can be combined with very
% high-resolution simulations of the Milky Way to provide a detailed
% evolutionary history of our Galactic home.

