%%%%
% -- Schedule and Budget breakdown and funding path
% --     FOBOS Keck White Paper 2019
%%%%

\section{Schedule, Budget, and Funding Path}
\label{sec:budget}

\subsection{Funding Path}

Early funding for FOBOS has been obtained through a number of
sources. First, conceptual development of Fiber-WFOS for TMT has
provided the backbone for the current FOBOS design. Second, a number
of smaller grants were used to develop and assess specific aspects of
FOBOS. Namely, a 2016 UCO mini-grant (\$XXk) to KG Lee was used for
conceptual development of the microlens optics; a 2017 UCO mini-grant
(\$XXk) to K. Bundy allowed for a study of sky-subtraction fidelity
in existing fiber systems; WMKO white-paper funds (\$40k) provided to
K. Bundy in 2018 are still being used to develop science cases, build
invested science teams, and perform focal-plane and spectrograph
design studies; and a 2018 UCO mini-grant (\$120k) to K. Bundy was
provided for design and fabrication of a microlens-coupled fiber
system primarily to demonstrate its throughput at Keck compared to
measurements in the lab. The latter builds on UCO's ongoing
investment in fiber testing equipment needed for a number of internal
projects has helped to further the FOBOS design. This Phase A funding
request is designed to bring the project to a stage of readiness
needed to submit proposals to the NSF MSIP (2020) and MsRI-2 (2023)
programs.

We intend to propose for FOBOS design funds via an MSIP solicitation
expected in early 2020; if successful, this will fund the full
instrument-design phase. The construction funding would then come
from a MsRI-2 proposal in 2023. We have requested ``Phase A'' funding
from late 2019 to mid 2021 to allow for continued development between
submitting our MSIP proposal and when the funding is made available.
In the event that we are unsuccessful in the MSIP proposal, our
requested ``Phase A'' funding in the last half of this proposal would
be used for preparation towards a MsRI-1 proposal in 2021. The
included schedule shows our NSF funding plan as funding windows at
the top of the Gantt chart.

We intend to apply for other smaller funding opportunities as they
become available, including the UCO mini-grant and NSF ATI grants.
Both of these would be targeted at relatively self-contained design
components of FOBOS's overall system. Other government funding and
private funding is also being pursued as opportunities become
available.

