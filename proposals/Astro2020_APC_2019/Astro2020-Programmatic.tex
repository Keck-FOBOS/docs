%%%%
% -- Proposed Work and Budget
% --     FOBOS Keck White Paper 2019
%%%%

% ========================================================================
% - leverage experience from Shane prime focus ADC; Harland design; Matt
%   Radovan may know; Dave Cowley probably best person to talk to

\subsection{Technology Drivers}
\label{sec:design}

FOBOS will provide key deseriables in the near-term thanks to deployment at the existing Keck II telescope.  While it carries a comparatively modest cost compared to future proposed facilities (e.g., MSE, SpecTel), it helps lay the groundwork for the realization of those facilities.



\noindent \textbf{Spectrograph Cost:} Here we make the case that FOBOS is a platform for figuring out how to build future dedicated spectroscopic telescopes like SpecTel cheaper.  With flexible focal plane deployment, FOBOS can serve as a platform for cost-cutting technology development.  Fixed-format instruments like PFS and DESI do not support such development.

\noindent \textbf{Starbugs fiber positioners:} Starbugs are a
positioning technology developed and deployed by the Australian
Astronomical Optics (AAO), which has partnered with our team to
generate a conceptual design for use of Starbugs by FOBOS. Starbugs
are currently being tested on-sky with the TAIPAN instrument at UK
Schmidt Telescope and published results on their performance are
expected in summer 2019. An AAO contract included in this request
(\$60k) will yield the performance details, quantitative risk
assessment, and project planning required for our proposal for full
design funding via NSF's MSIP program. During a recent visit to UCO,
Jon Lawrence, AAO's Head of Technology, agreed to support the rapid
MSIP proposal schedule.


\noindent \textbf{Data Systems:} We will develop the requirements and
initial concept for the FOBOS data simulator following instrument
forward-modeling techniques developed by DESI. The simulator will
enable tests of potential FOBOS science cases (e.g., exposure time
estimates, redshifting success). For the MSIP proposal, we will
construct a detailed plan for evolving the data simulator into
data-reduction and data-analysis pipelines. The FOBOS data-analysis
pipeline (DAP) will take advantage of the fixed spectral format and
common target classes to provide high-level data products, including
Doppler shifts, emission-line strengths, and template continuum fits
(cf., Westfall et al.; SDSS-IV MaNGA DAP). Planning will include
development of user-friendly platforms built on the Keck Observatory
Archive for serving raw data, reduced spectra, and DAP science
products.


\subsection{Current Status}

FOBOS has been awarded Phase-A funding by WMKO Observatory and received the full endorsement of the Keck Science Steering Committee....


\subsection{Cost Estimates}

% \noindent \textbf{Operations:} Powered by Starbugs fiber positioners,
% FOBOS will enable fast ($<$2 minute), dynamic reallocation of fibers. We will
% develop an initial target allocation simulator to determine
% efficiencies for various science programs and explore options for
% program combination and optimization under different observing
% scenarios. This work requires planning interfaces with the fiber
% positioning control software and the Keck user.


% To efficiently
% determine the best options given a wide range of possible targets and
% desired observing outcomes, we will develop a conceptual design for
% MAISTRO,\footnote{MAISTRO: Modular Artificial Intelligence System for
% Target Reallocation and Observing.} an ``artificial intelligence''
% (AI) targeting system that will learn optimization strategies for
% assigning targets from a database of overlapping observing programs
% with pre-defined priorities. The AI package will aggregate data
% quality using a quick-look reduction package, science-driven
% performance metrics, {\it and real-time assessments of the observing
% conditions} to make dynamic targeting recommendations. For example,
% if conditions are slightly less than optimal, MAISTRO would
% reconfigure Starbugs to brighter objects in a field or implement a
% different program prioritization. MAISTRO will incorporate updated
% target lists and priorities from the active observer and could easily
% be over-ridden at any time. Fractions of the full FOBOS multiplex
% might also be reserved ``manual targeting'' as required by the
% program PI.

%   - maintains a database with observational progress on individual
%     targets in the survey and
%   - dynamically reallocates fibers based on real-time assessments of
%     the aggregate S/N of each target to meet the specific need of each
%     science case.

% This requires significant design and testing of a combined software
% package and hardware interface.  Specific considerations involve (1)
% fast and robust reduction procedures (cf. MaNGA DOS) that can assess
% the aggregate data and (2) a responsive database with a schema
% optimized for real-time decision making to select targets for
% (re)acquisition while accounting for collision limitations.  Provided
% enough design effort, this lends itself to a machine-learning
% application.

