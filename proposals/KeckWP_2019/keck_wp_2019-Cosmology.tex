%%%%
% -- Cosmology Science
% --     FOBOS Keck White Paper 2019
%%%%

\subsection{Enhancing Dark Energy Probes via Precision Cosmic Distances}
\label{sec:cosmology}

\noindent \textbf{Photo-$z$s} Enormous world-wide efforts --- culminating in
LSST, Euclid, and WFIRST --- are seeking highly precise measures of cosmic
structure to constrain the evolving dark-energy equation-of-state. These measures utilize angular correlations of galaxy positions, their
gravitational lensing shear, and the cross-correlation between the two.
Unfortunately, photometric distances (via photometric redshifts, or
``photo-$z$s'') are significantly less precise than spectroscopic
redshifts (spec-$z$s), introducing significant biases.  FOBOS offers
spectroscopic validation of photo-$z$s that is 
therefore critical to the success of {\it all} imaging surveys in this
respect. It would not only \emph{increase the dark energy
figure-of-merit in LSST by 40\%} \citep{newman15} but, importantly,
provide vital confidence in cosmological results.  FOBOS is particularly
powerful in this application because it has no ``redshift desert'' thanks to its unique ability to measure spectroscopic redshifts above $z > 1.5$ via
rest-frame UV features.  This eliminates the need for expensive, space-based\footnote{Ground-based near-IR
spectroscopy is too contaminated by sky-line emission to provide spec-$z$s at the required level of completeness
\citep{newman15}.} near-IR spectroscopy.

\noindent\comment{LBG cosmology: Wilson, White?}

\noindent\comment{Kinematic Weak Lensing: Bundy, Huff, Schlegel, DiGiorgio?}