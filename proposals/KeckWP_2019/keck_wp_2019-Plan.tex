%%%%
% -- Proposed Work and Budget
% --     FOBOS Keck White Paper 2019
%%%%

\section{Proposed Work}
\label{sec:design}

The immediate need for FOBOS development is design funding that leads to
compelling proposals for significant investiment from the NSF and/or a
private foundation. We outline our funding strategy \comment{somewhere},
and we make the following three requests of WMKO and the Keck SSC:
\textbf{
\begin{enumerate}
\item We request approval to pursue our funding path at the level needed
to fully fund FOBOS through design, construction, integration, and
commissioning.
% I think we want input from Mark and John on this fist item.  I could
% see them balking at permission through construction and
% commissioning-NKM.  
\item We request seed funding of \comment{xx} over the next two years
(FY 2020 and 2021) needed to advance the instrument design to the level
needed to prepare competitive proposals for upcoming NSF MSIP and MsRI
solicitations.
\item We request engineering support from the WMKO staff for help
developing interface systems to the FOBOS conceptual design.  The
necessary tasks list for WMKO effort has been iterated on between the
FOBOS team and the Observatory in preparation for the MsRI-1
pre-proposal submitted earlier this year. These tasks are detailed in
the included schedule.
\end{enumerate}}

\subsection{Instrument Design Work and Project Planning} We list below
the primary instrument-design and project planning activities that we
have identified as top priorities for FOBOS development in the lead up
to larger funding requests. The schedule and budget for these activities
are outlined in Section \comment{additional pages}.  It is worth noting
that engineering effort for this development phase will come from both
UCO and Space Sciences Lab (SSL) at UC Berkeley.  Staff at SSL working
on the DESI project are becoming available and have interest in FOBOS
and expertise from DESI which significantly strengthen the FOBOS team.

\noindent \textbf{Projet Preparation} Continued development of the
schedule, budget, project execution plan, and systems engineering
structure needed for compliance with the NSF Major Facilities Guide.
This work will focus on preliminary and final design phases in
preparation for a MSIP design proposal followed by early planning for
construction needed for a future MsRI-2 proposal. 

\noindent \textbf{Atmospheric Dispersion Compensator (ADC):} Continued
optical work with a goal of pushing the current 17~arcmin field-of-view
to the full 20~arcmin field of view available at Keck.  The
optomechanical design of the lens cells.  Further development of the
motion systems for ADC articulation and field rotation.

\noindent \textbf{Focal Plane System:} Continued mechanical design;
fiber and Starbug actuator support, interface to front end module, and
interface to storage position. This system also defines one of the
interfaces to the Keck II Telescope and must comply with WMKO space
envelopes, servicing needs, and other requirements. The focal plane
system also inputs the guide cameras. 

\noindent \textbf{Starbugs fiber positioners:} Starbugs are a
positioning technology developed and deployed by the Australian
Astronomical Observatory (AAO), which has partnered with our team to
generate a conceptual design for use of Starbugs by FOBOS.  Design
requirements for Starbugs in FOBOS are more relaxed than the currently
on-sky TAIPAN instrument thanks to the larger physical plate scale at
Keck. A contact for further development of the Starbug system for Keck
is included as part of this funding request (\$60k).  This work package
will focus on conceptual design of the robotic focal plane including
optical feed back loop development, performance evaluation of the
Starbugs, AAO work package planning in preparation of the upcoming MSIP
proposal and reliability/risk mitigation of the AAO work package.

\noindent \textbf{Fiber System:} Out side of this funding request a UCO
mini-grains{\it Fiddles} is under development to specifically retire
risks associated with the micro-optics at the focal plane.  This work is
on-going and will be a major step forward in developing this sub-system.
Work on the fiber system under this proposal will focus on system level
design of the fiber system including cabeling and cable support. Further
development of the slit input side of the fiber system and development
of the micro-lens array for a central, fixed-position 4.5~arcsec
diameter IFU for fast target of opportunity acquisition.

\noindent \textbf{Spectrographs.} The conceptual design of the
spectrograph comes from work developed for Fiber-WFOS.  Continuing work
on the spectrograph will focus on documentation and modifications needed
in preparation for the MSIP proposal. Specifically answering questions
of risk reduction with the camera systems.

% Put the calibration system back in? Yes-NKM
\noindent \textbf{Calibration System.} Given the need for a dome screen
and lamp projection system the calibration system is a joint development
between WMKO and the FOBOS team.  The work in this proposal will focus
on defining the calibration requirements followed by conceptual design
development of the screen and projection system in partnership with the
WMKO engineering staff.

% \noindent \textbf{Auxiliary Systems.} Design of auxiliary systems
% inputs Nasmyth platform interfaces, utilities access, fiber routing
% and support, thermal control and vibration control systems.

\subsection{MAISTRO: Target Allocation with Artificial Intelligence}
\label{sec:targeting}

\comment{edit/remove this?} Powered by Starbugs fiber positioners, FOBOS
will enable fast, dynamic reallocation of fibers. To efficiently
determine the best options given a wide range of possible targets and
desired observing outcomes, we will develop a preliminary design for
MAISTRO,\footnote{MAISTRO: Modular Artificial Intelligence System for
Target Reallocation and Observing.} an ``artificial intelligence''
(AI) targeting system that will learn optimization strategies for
assigning targets from a database of overlapping observing programs
with pre-defined priorities. The AI package will aggregate data
quality using a quick-look reduction package, science-driven
performance metrics, {\it and real-time assessments of the observing
conditions} to make dynamic targeting recommendations. For example,
if conditions are slightly less than optimal, MAISTRO would
reconfigure Starbugs to brighter objects in a field or implement a
different program prioritization. MAISTRO will incorporate updated
target lists and priorities from the active observer and could easily
be over-ridden at any time. Fractions of the full FOBOS multiplex
might also be reserved ``manual targeting'' as required by the
program PI.

%   - maintains a database with observational progress on individual
%     targets in the survey and
%   - dynamically reallocates fibers based on real-time assessments of
%     the aggregate S/N of each target to meet the specific need of each
%     science case.

% This requires significant design and testing of a combined software
% package and hardware interface.  Specific considerations involve (1)
% fast and robust reduction procedures (cf. MaNGA DOS) that can assess
% the aggregate data and (2) a responsive database with a schema
% optimized for real-time decision making to select targets for
% (re)acquisition while accounting for collision limitations.  Provided
% enough design effort, this lends itself to a machine-learning
% application.

\subsection{Automated Data Products}
\label{sec:DAP}

\comment{edit/remove this?} While the FOBOS data simulator is required for
our data-science challenges, it also forms the basis of a delivered
data reduction pipeline (DRP) for this instrument. This software will
provide both the quick reduction assessments needed for dynamic
targeting, as well as full reductions for scientific analysis. In the
proposal period, we will also develop a preliminary design for a data
analysis pipeline (DAP). Unique among Keck instruments, the FOBOS DAP
will take advantage of the fixed spectral format and common target
classes to provide high-level data products, including Doppler shift,
emission-line strengths, and template continuum fits (cf., Westfall
et al.; SDSS-IV MaNGA DAP). The DAP will also produce results from
relevant machine-learning applications (e.g., redshifts at low-S/N).

Raw data, reduced spectra, and high-level DAP science products will
be publicly delivered via user-friendly platforms built on the Keck
Observatory Archive. After associated proprietary periods, data will
be served for {\it all} FOBOS observations, creating a rich legacy
data set for the astronomical community. Both program PIs and the
larger community will be encouraged to develop the DRP and DAP to
meet the needs of specific science applications. These software
packages will be open source and publicly served (e.g., using
GitHub).
