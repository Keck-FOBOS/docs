%\documentclass[11pt,letterpaper]{article}
\documentclass[oneside,11pt]{amsart}

%\usepackage{a4wide}
%\usepackage{epsfig}
%\usepackage{psfig}
\usepackage{graphicx}
\usepackage{natbib,latexsym,url,enumitem,pdfpages}
\usepackage{color}
\usepackage{wrapfig}
\usepackage{caption}

\captionsetup{
    justification=justified,
    margin=0pt,
    font=small}


\newcommand{\arcsec}{\mbox{$^{\prime\prime}$}}
\newcommand{\gt}{$>$}

% Some fancy commenting
\definecolor{todo}{RGB}{200,0,0}
\newcommand{\comment}[2][todo]{{\color{#1}[[{\bf #2}]]}}

% Challenge counter
\newcounter{chalno}
\newcommand{\chal}[1]{\refstepcounter{chalno}\label{#1}}

% User commands
\input{journaldefs}

\DeclareRobustCommand{\gtrsim}{%
\mathrel{\hskip-.5em\begin{array}{c}>\\[-8pt]\sim\end{array}\hskip-.5em}}
\DeclareRobustCommand{\lesssim}{%
\mathrel{\hskip-.5em\begin{array}{c}<\\[-8pt]\sim\end{array}\hskip-.5em}}



\pretolerance=10000
\textwidth=6.4in
\textheight=8.95in
\voffset = 0.in
%\voffset = -0.3in  % For my printer
\topmargin=0.0in
\headheight=0.00in
\hoffset = 0.0in
%\hoffset = 0.33in  %  For my printer
\headsep=0.00in
\oddsidemargin=0in
\evensidemargin=0in
\parindent=2em
\parskip=0.2ex
 
\renewcommand{\baselinestretch}{1.03}

\special{papersize=8.5in,11in}

\newcommand{\markus}{\textcolor{green}}

\setlength{\parskip}{0.6 ex plus 0.4ex minus 0.2ex} \flushbottom
\pagestyle{plain} 

\begin{document}
% \thispagestyle{empty}

\pagenumbering{arabic}

\vspace*{-1.5cm}

\centerline{\textsf {\Large The FOBOS Spectroscopic Facility for Keck: Project Description}}


% \centerline{\textsf {\large Project Summary}}

% \bigskip
% \noindent {\bf Overview:} In this \emph{Design}
% submission, we propose to dramatically enhance the power of upcoming
% panoramic deep-imaging from the Large Synoptic Survey Telescope (LSST),
% Euclid and the Wide-Field Infrared Survey Telescope (WFIRST) in order to
% address key questions in the areas of dark energy, the galaxy ecosystem
% at $z\sim2$, and the assembly history of the Milky Way and Local Group
% Galaxies.  We will design the astrostatistics, instrumentation, and
% software solutions required over the next decade to provide optimized
% spectroscopic training sets that can unlock \emph{physical
% information} (e.g., redshifts, galaxy star formation histories, stellar
% metallicities) from deep photometry alone.  Applying machine learning to
% a set of ambitious Data Science Challenges using simulated data, we will define
% requirements on future spectroscopic training sets.  These requirements
% will guide the preliminary design of FOBOS, a powerful new spectrograph
% to deploy in 2026 on the 10 m Keck II Telescope.  FOBOS will provide
% publicly-available deep, high-multiplex spectroscopy with high target
% sampling and flexibility uniquely matched to the ``Big Data'' training
% problem.

% \bigskip
% \noindent {\bf Intellectual Merit:} High-multiplex and deep
% spectroscopic followup of LSST and other panoramic deep-imaging surveys
% is a widely recognized necessity.  Reports in 2015 and 2016 by the
% National Research Council and National Optical Astronomical Observatory
% specifically recommend that the NSF support construction of required
% spectroscopic facilities because none currently exist or are planned at
% U.S.\ observatories.  FOBOS satisfies these spectroscopic needs at
% relatively low cost by utilizing the existing 10 m Keck II Telescope, a
% highly-successful U.S.-led large telescope.  Even with the powerful capabilities of FOBOS deployed, the astronomical
% community recognizes the need for cutting-edge data science techniques to ``train''
% vast photometric surveys with what will necessarily be more limited
% spectroscopy.  Success in the training methodologies we propose here
% will make photometric redshifts more precise, improving the LSST dark
% energy figure-of-merit by 40\%.  They will enable a comprehensive
% understanding of galaxies and their gaseous environments at $z\sim2$,
% and they will reveal fossilized structures in the Milky Way, M31, and
% other Local Group galaxies through chemical signatures inferred for
% millions of stars.

% \bigskip
% \noindent {\bf Broader Impacts:} We will build on the success of several ongoing programs at UC Santa Cruz (UCSC) that
% connect high school and college students, especially those from underrepresented minorities, to active research groups.
% Studies show that such connections increase STEM retention.  The flagship program is Akamai, run by UCSC's Institute
% for Scientist and Engineer Educators (ISEE), which advances college students from Hawai'i into the STEM workforce.  Our
% proposal supports two Akamai interns to work at UCSC on instrument simulation and design as well as machine learning
% for spectroscopic analysis.  We will also engage a cohort of graduate students in ISEE's Professional Development
% Program which builds teaching skills as students develop an inquiry-based activity. The graduate students will then
% conduct this activity, centered on FOBOS instrument development, with 25 largely underrepresented community college
% students from UCSC's Lamat program.  Finally, we will take advantage of a successful hands-on research course and a
% long-running summer internship program to introduce data simulation and machine-learning techniques to 1st-year
% undergraduate and senior high school students.

% \clearpage


\setcounter{page}{1}

\centerline{{\it MSIP proposal category: ``Development Investments''}}

\section{Scientific Justification} 

\subsection{Introduction}
Led by NSF's LSST\footnote{
%
LSST: Large Synoptic Survey Telescope.  LSST will be begin science operations in 2023.}
%
and NASA-supported missions like Euclid\footnote{
%
Euclid is led by the European Space Agency with significant NASA involvement and will launch
in 2021. Its primary mission is a 15,000 deg$^2$ imaging survey in optical and near-IR wavebands.} 
%
and WFIRST\footnote{
%
WFIRST: The Wide Field Infrared Survey Telescope, expected to launch in the mid 2020's.},
%
astronomy is entering a new era of unprecedented deep-imaging campaigns that will survey huge volumes of the
Universe.  From the emergence of the earliest galaxies from a ``primordial soup'' of gas and dust, to the peak of
cosmic star formation and the current era of accelerated expansion, these surveys will provide unprecedented statistics
at key epochs of cosmic history.

% Meanwhile, the rate of cosmic expansion was beginning to accelerate,
% as the Universe became increasingly dominated by ``Dark Energy,''
% whose origin remains the single greatest mystery in astronomy and
% cosmology today.

% Since Edwin Hubble's observations over 100 years ago,

Even so, gaining physical insight from panoramic imaging surveys will require intensive spectroscopic follow-up.  The
power of combining photometry and dedicated spectroscopy is widely appreciated and perhaps best illustrated by the
success of the Sloan Digital Sky Survey (SDSS) which used this combination to record the properties of over 1 million
galaxies, mapping the present-day universe and making SDSS one of the most highly cited surveys in the history of
astronomy.

% Because a quality spectrum requires far more observing time per source
% than an image, SDSS pioneered ``high multiplex'' spectrographs,
% capable of \emph{simultaneous} spectroscopy of hundreds of objects.

LSST's all-sky images, for example, will be 1,000 times deeper and detect far more
distant galaxies than SDSS, but \textbf{no current U.S.~facility is
capable of obtaining spectroscopic follow-up of LSST galaxies} at a level
required to capitalize on the \$1B the U.S.\ has invested in that
project.  In fact, an SDSS-like spectroscopic study of 1 million
galaxies at LSST depth would require 300 years of observing on the
largest telescopes with current instrumentation!  

This proposal addresses this challenge through the design of an ambitious spectroscopy facility on one of the world's largest telescopes.  Timed to deploy on WMKO's\footnote{WMKO: W.~M.\ Keck Observatory operates the two twin 10m Keck Telescopes.} Keck II Telescope in 2028, just as various panoramic deep imaging surveys hit their stride, FOBOS's\footnote{FOBOS: Fiber-Optic Broadband Optical Spectrograph} 1800-fiber multiplex, deep sensitivity, high sampling density, and wide wavelength coverage is optimized for the \emph{deep-drilling} spectroscopic training sets required to extract maximum information from wide-field photometry.  Its flexible target allocation system and multiplexed IFU mode provide unique capabilities for realizing major progress on fundamental goals in Cosmology, Galaxy Formation, and Local Group Archaeology, and Time Domain Astronomy in the coming decade.

\subsection{Community Benefits}

Based on science requirements derived from reference-mission key programs, we present a comprehensive plan for optimizing and completing the design of the FOBOS instrumentation, calibration and support systems, operational and planning software, and data management systems.  We propose community engagement activities to coordinate involvement of U.S.~astronomers including those outside the Keck Community in developing and leading FOBOS public survey programs whose data products will benefit wide swaths of the astronomical community.  In partnership with NOAO (this name might change), we will develop additional ``open-access'' models to offer $\sim$100,000 fiber-hours per year in support of individual, PI-led programs to be integrated into the FOBOS observing suite.  Finally, we have emphasized the design of software platforms necessary for a seamless user experience from target submission to data product retrieval and analysis.  FOBOS will be the first general-purpose spectroscopic instrument to automatically provide high-level data products such as redshifts, stellar continuum fits, and emission line measurements.  With a commitment to the public release of such products derived from \emph{all} FOBOS observations, these data products will dramatically reduce the time from observations to science.

As part of the work we propose here, NSF's OIR Lab will solicit additional key program concepts from the U.S. community, host workshops to discuss and refine these concepts, and coordinate proposing teams ahead of a competed selection process to define design-reference programs in support of Preliminary and Final Design.

% \subsection{Research Community Priority} 
% \label{sec:community}

% The need for spectroscopic follow-up in the LSST era was made clear in
% the National Research Council's 2015 report, ``Optimizing the U.S.
% Ground-Based Optical and Infrared Astronomy System'' \citep{NAP21722}:
% %
% \noindent\begin{center}\mbox{\parbox{0.95\linewidth}{
% %
% The National Science Foundation should support the development of a
% wide-field, highly multiplexed spectroscopic capability on a medium- or
% large-aperture telescope in the Southern Hemisphere to enable a wide
% variety of science, including follow-up spectroscopy of Large Synoptic
% Survey Telescope targets. Examples of enabled science are studies of
% cosmology, galaxy evolution, quasars, and the Milky Way.
% %
% }}\end{center}

% Workshops organized by the National Optical Astronomy Observatory (NOAO)
% in 2013 and 2016, the latter at the NSF's request, reported specific
% spectroscopic needs for LSST follow-up in all science areas.  In
% particular, the 2016 report notes that a critical resource in need of
% prompt development is to ``Develop or obtain access to a highly
% multiplexed, wide-field optical multi-object spectroscopic capability on
% an 8m-class telescope.''  Based on these recommendations, we propose the
% FOBOS instrument coupled with a suite of data-driven tools to address
% the spectroscopic requirements of LSST and other photometric surveys at
% a final cost 20 times less than a new Southern Hemisphere facility.
% Located in Hawaii, FOBOS can access more than 70\% of the LSST
% footprint, more than adequate for building powerful
% spectroscopic training sets.  Compared to the Prime Focus Spectrograph
% (PFS) on Japan's Subaru Telescope, FOBOS would be 1.7$\times$ faster,
% provide unique UV sensitivity (0.31--1 $\mu$m compared to
% 0.38--1.25 $\mu$m with PFS), and offer higher-density, more flexible
% target sampling over ``deep-drilling'' fields.  Unlike PFS, FOBOS would be operated
% on a U.S.\ telescope with dedicated U.S.\ access and a commitment to
% supporting U.S.-led imaging facilities.  FOBOS is also complementary to
% future telescopes and instruments that would be optimized to cover wider areas
% (several deg$^2$ per pointing) at shallower depths.

%\comment{mention FOBOS can do PI-led science too}

% The need for deep spectroscopic follow-up is particularly acute for the major cosmological probes to be carried out by
% LSST, Euclid, and WFIRST, which all rely on ``photometric redshifts:'' measures of galaxy redshift, $z$
% --- a direct proxy of distance and look-back time---based on imaging alone. \citet{newman15} summarize the case for a
%     significant spectroscopic campaign to calibrate and train LSST photometric redshifts in order to improve cosmological constraints.  They describe a redshift survey that,
%     if carried out with FOBOS, would increase LSST's Dark Energy figure-of-merit by 40\% at a cost of less than 5\% of
%     the LSST budget.  The urgent case for spectroscopic redshift training has been the subject of numerous publications
%     \citep[e.g.,][]{laureijs11, masters15, hemmati18}.

% Meanwhile, the astronomy community recognizes that the coming ``Big
% Data'' era, culminating in LSST, necessitates ``\textbf{harnessing the
% data revolution}.''  Widespread community interest in advanced
% data-science techniques continues to grow amidst calls for educational
% programs, conference series, and research funding to support the growth
% of a new field, ``Astroinformatics,'' which exploits the interface
% between astrophysics and statistics \citep{borne09}.  Astronomy's
% largest organizations, including the American Astronomical Society and
% the International Astronomical Union, have supported active working
% groups on astroinformatics and astrostatistics since 2015.  LSST itself
% has supported the Informatics and Statistics Science Collaboration and
% partnered with NSF on the Data Science Fellowship Program to train
% astronomy graduate students in data-science techniques.  Our proposal
% builds on and contributes to these ongoing efforts.

\subsection{Key Science Goals}
\label{sec:goals}

We present three ``design-reference'' programs that capture FOBOS's primary science goals and drive initial requirements.  These programs highlight FOBOS's capabilities in addressing the nature of Dark Energy, the formation of galaxies, and the assembly history of the Andromedra Galaxy system.  


% \comment{modify the motivation here to reflect our funding request;
% move away from being LSST centric}

% \begin{figure}[h!]
% %
% \vskip -0.1in
% %
% \includegraphics[width=\textwidth]{figs/Hemmati18_Fig8_VVDS_spec.png}
% %
% \caption{\small {\it Left}: A Self-Organizing Map
% \citep[SOM;][]{1990Natur.346...24K} from \citet{hemmati18} encoding the
% relation between colors in an LSST+WFIRST-like color space and redshift,
% $z$.  Position in the SOM is associated with a position in the
% multi-dimensional broad-band color space of galaxies.  Galaxies observed
% in this space are assigned $z$ values based on the median photo-$z$ of
% galaxies from the CANDELS survey \citep[color
% bar;][]{2011ApJS..197...35G}.  Such SOMs can be used to optimally define
% spectroscopic training samples for use with imaging surveys.  {\it
% Right}: Galaxy spectra from VVDS \citep{2005A&A...439..845L}; black
% crosses near the top and bottom of the SOM are plotted in the top and
% bottom panels, respectively.  Note the similarity of the high-resolution
% spectra associated within the SOM, suggesting that a systematic
% spectroscopic exploration of the LSST color space would have
% far-reaching benefits to the science return of the mission beyond the
% photo-$z$ application.}
% %
% \label{fig:SOM}
% %
% \end{figure}

%-----------------------------------------------------------------------
\subsubsection{Enhancing Dark Energy Probes via Precision Cosmic Distances}
\label{sec:cosmology}

The quest to understand Dark Energy has motivated billions of dollars of investment in efforts world-wide ---
culminating in LSST, Euclid, and WFIRST --- that seek highly precise measures of cosmic structure to constrain the
evolving dark-energy equation-of-state. Delineating cosmic structure requires measurements of galaxy positions and
gravitational shear as a function of distance over vast cosmic volumes.  For the billions of sources that will be
cataloged, distances must be derived from photometric redshifts (photo-$z$s), whose poor accuracy (among other
challenges) can introduce significant biases in cosmological results.

The FOBOS Cosmology Program will play a critical role by training photo-$z$s from sources that are too faint for other
instruments (like PFS) but will dominate the number of galaxies and cosmic volume probed by panoramic deep imaging in
the next decade.  Complete spectroscopic training to $i_{AB} = 25.3$ will \emph{increase the LSST's dark energy
figure-of-merit by 40\%} \citep{newman15}. FOBOS's strength in this endeavor goes beyond sensitivity.  With no
``redshift desert,'' thanks to its unique ability to measure spectroscopic redshifts above $z > 1.5$ via rest-frame UV
features, the FOBOS Cosmology Program will dramatically reduce the need for expensive,
space-based\footnote{Ground-based near-IR spectroscopy is too contaminated by sky-line emission to provide spec-$z$s at
the required level of completeness \citep{newman15}.} near-IR spectroscopy.

\chal{photozs}
%
\begin{enumerate}[rightmargin=0.2cm,leftmargin=0.2cm]
%
\item[] {\textsf {\large FOBOS Cosmology Program:}}  For a set of 12 0.1 deg$^2$ FOBOS pointings arranged evenly in longitude and chosen to overlap with LSST, Euclid, and WFIRST footprints, this program will execute ultra-deep integrations of $\sim$15,000 $24 < i_{AB} < 25.3$ sources using 1200 single fibers (per pointing) from two of FOBOS's three spectrographs.  Fiber-IFUs from the 3rd spectrograph will be used for simultaneous Galaxy Program observations (see below).  To maximize the power of the resulting photo-$z$ training sample, we will prioritize rare targets in color-magnitude space, e.g., following \citet{masters19}, and will dynamically re-allocate fibers to new targets as successful redshifts are obtained (i.e., some fibers will deliver multiple spec-$z$s).  The longest integration times per field will reach 100 hours.  Accounting for two-thirds of available fiber-hours, this program would require 23 effective dark nights per year for 5 years. %34 total

\end{enumerate}

%-----------------------------------------------------------------------
% \begin{wrapfigure}{r}{0.5\textwidth}\small
% %
% \includegraphics[width=0.5\textwidth]{figs/Parks_fig.pdf}
% %
% \caption{Application of machine learning to find and quantify the
% physical parameters of absorption by neutral hydrogen gas in spectra
% taken along quasar sight lines \citep[adapted from Figs 7 and 14
% from][]{parks18}.  Two absorption systems in the spectrum (top-left) are
% identified (middle-left) and then ``labeled'' with an HI column density
% ($N_{\rm HI}$) (bottom-left) using a convolutional neural network (CNN).
% Redshift, $z$, (top-right) and $N_{\rm HI}$ (bottom-right) measurements
% obtained the CNN are in excellent agreement with derivations by experts.
% FOBOS will provide rich data sets for similar transfer of
% physical parameter labels to photometric and spectroscopic data.}
% %
% \label{fig:absorber}
% %
% \end{wrapfigure}

\subsubsection{Mapping the Baryonic Ecosystem of Early Galaxies at All Scales}
\label{sec:galaxies}


The fueling and regulation of galaxy growth during the peak formation epoch ($z \sim2$--3) is critically tied to the turbulent and gas-rich ecosystem in which early galaxies evolve.  James Webb Space Telescope and Extremely Large Telescopes will marshal powerful infrared observations to study the stars and nebular gas at the heart of these early galaxies.  But to map out their crucial link to the extended gas reservoirs, diffuse halos, and streaming filaments that dominate the mass in these environments requires an instrument like FOBOS.  Its deep sensitivity and high sampling density enables comprehensive tomographic reconstruction of the intergalactic medium (IGM) across the largest cosmic structures in a single pointing ($\sim$10 transverse Mpc at $z \sim 2.5$).  Its blue sensitivity probes Ly-$\alpha$ across the complete formation epoch ($z = 1.5$--3.5) and opens access to high-ionization transitions that reveal diffuse gas \emph{in emission}, such as O VI (1032 \AA).  Finally, its ability to combine single-fiber and multiplexed IFU observations allows us to map the density and dynamical state of diffuse gas at all relevant scales from the IGM to the circumgalactic medium (CGM).


% Billion galaxy survey?

\begin{enumerate}[rightmargin=0.2cm,leftmargin=0.2cm]
%
\chal{phot}
%
\item[] {\textsf {\large FOBOS Galaxy Ecosystem Program:}} This program consists of two parts that integrate into and build upon the FOBOS Cosmology Program.  Additional IGM mapping of the 12 Cosmology pointings will expand the sampled area of each field to 0.5 deg$^2$ (5 pointings each) at a depth of 3 hours.  This depth ensures sufficient Ly-$\alpha$ absorption mapping with a sightline density of $\sim$1600 per FOBOS pointing across absorber redshifts, $z = 1.5$--3.5 \citep[see][]{lee16}.  Field selection will take advantage of quasar-quasar pairs at certain redshifts.  The number of fields and mapped area will capture the largest modes of large-scale structure while beating down sample (``cosmic'') variance.  This program component requires a total of 26 nights, spread over 5 years.

The second component of this program runs in parallel with the ultra-deep Cosmology Program.  With two-thirds of fibers allocated to photo-$z$ training in that program, the remaining third will be configured into 16 fiber IFUs, each composed of 37 fibers spanning 5.6 arcsec.  With 100-hour integrations, these will map emission lines from the CGM on 5 kpc scales out to $r < 23$ kpc for $z \sim 2$ galaxies spanning a range of $M_*$ and SFR.  The final sample of nearly 200 direct CGM maps will link the buildup of the CGM through heating and gas flows to the cosmic web on large scales (mapped via IGM tomography) as well as the internal structure of the galaxies themselves (as observed by JWST and and at high resolution by ELT instruments).  This component utilizes the equivalent of 11 nights of dark time per year for 5 years.

%
\end{enumerate}


%-----------------------------------------------------------------------
\subsubsection{Assembly of Andromeda's Disk and Satellite Galaxies}
\label{sec:localgroup}

Facilities like Gaia and APOGEE are probing the assembly history of the Milky Way and its halo by mapping their
chemo-dynamical structure in unprecedented detail.  In the next decade, a major goal is to building toward this level
of understanding for the Andromeda Galaxy, its halo, and its satellite galaxies.  Instruments like PFS with
degree-scale fields of view that span 20 kpc or more at M31 distances are well suited to characterizing M31 halo
structure.  But FOBOS's much higher sampling density is critical for breakthrough data sets of stellar tracers in the
M31 disk, M33, and Andromeda's major satellites.  High S/N FOBOS spectra will map patterns of [Fe/H] and [Mg/Fe] and
link these chemical tags to the underlying dynamics with unprecedented statistical power.  Combining these FOBOS
observations with integral-field data from the SDSS-V Local Volume Mapper and PFS surveys of halo structure, a complete
picture of the Andromeda system's formation history will address key questions about disk evolution, dwarf galaxies,
and substructure that have so far been limited to Milky Way and its surroundings.




\begin{enumerate}[rightmargin=0.2cm,leftmargin=0.2cm]

\chal{stellar} 
%
\item[] {\textsf {\large FOBOS Andromeda Program:}} This program will survey 100,000 primarily RGB stars in the M31 disk, 10,000 stars in M33, and a further 15,000 stars in the central regions of Andromeda's major satellites: NGC 185, NGC 147, and And II.  Accounting for a 60\% rejection rate \citep[see][]{dorman12} due to crowding of ground-based RGB catalogs ($i_{Vega} < 22.5$), on-sky densities of 6 pointings in the PHAT\footnote{Describe and reference PHAT.} region between 5--20 kpc average 1 isolated source every 10\arcsec{}, suitable for 10 independent fiber assignments per pointing.  Two disk pointings beyond 20 kpc will be visited once.  With order-of-magnitude longer integrations than SPLASH\footnote{Describe and reference SPLASH.}, total integrations of 10 hours per visit provide ${\rm S/N} \approx 20$ per resolution element for the faintest targets, a spectral quality sufficient for [Fe/H] with 0.1 dex precision and kinematic fits.  For the ``sweet-spot'' of RGB tracers at $i_{Vega} = 21.5$, ${\rm S/N} \approx 50$, enabling [Mg/Fe] with 0.1 dex precision.  Outside of M31, 6 10-hour visits will cover M33 and 3 visits each are assigned to NGC 185, NGC 147, and And II.  This program requires 23 nights per year for 5 years.  




\end{enumerate}

%-----------------------------------------------------------------------
%-----------------------------------------------------------------------
\section{Technical Overview}
\label{sec:project}

% \noindent \comment{1 page}

% Here's an alternative way to put in figures if we want captions on the side (to save space)
% Could introduce a new ``counter'' to count and label figures appropriately
%\centerline{\hbox{\includegraphics[width=0.6\textwidth, angle=0]{figs/FOBOSatKeck_v1.pdf}
%    \hspace{0.1cm} \vspace{2in}
%    \parbox[b]{0.3\textwidth}{\small {\bf Figure ??:} Rendering of FOBOS instrument systems deployed at the Keck II Nasmyth port.  By mounting the FOBOS spectrographs under the Nasmyth platform, other instruments like DEIMOS can maintain access to the telescope. \vspace{2cm}}}}

\begin{figure}[h!]
%
\vskip -0.1in
%
\includegraphics[width=\textwidth]{figs/FOBOS_inst.pdf}
%
\caption{\small {\it Left}: Rendering of FOBOS instrument systems
deployed at the Keck II Nasmyth port.  By mounting the FOBOS
spectrographs under the Nasmyth platform, other instruments like DEIMOS
can maintain access to the telescope. {\it Right}: Rendering of one of
the three four-armed FOBOS spectrographs.}
%
\label{fig:layout}
%
\end{figure}

\noindent \textbf{Focal Plane System.} Mounted at the Nasmyth focus of Keck II Telescope at WMKO (Fig \ref{fig:layout}), FOBOS is a modular instrument
composed of several major system.  Telescope light passes through the first two 946 mm diameter lenses of a 3-lens
compensating lateral atmospheric dispersion corrector (CLADC) before encountering a 45$^\circ$ mirror that folds the
beam vertically upwards.  The horizontal 3rd lens of the CLADC is positioned at focus above the fold mirror and serves
as the mounting plate for a complement of Starbugs that roam its upper surface, positioning downward-pointing fibers on
desired targets.  The risk of Starbugs adhesion loss and drops is significantly reduced by the fixed horizontal focal
surface, which preserves a normal gravity vector for the Starbugs and, because they are mounted on top, reduces each
bugs' vacuum adhesion requirements.  The focal surface rotates as the telescope tracks.  Starbugs can patrol zones up to several arcminutes and can be placed as close as 10\arcsec{}.  Back illumination combined with a fast, imaging metrology camera enables reconfiguration times down to 2 minutes. 

\noindent \textbf{Spectrographs.} The stress-relieving fiber run to FOBOS's three adjacent spectrographs is kept short ($< 10$m) to
preserve throughput at UV wavelengths.  Each spectrograph uses dichroics to divide the 140 mm diameter collimated beam
into four wavelength channels with combined, instantaneous coverage from 0.31--1 $\mu$m.  Fused-silica etched (FSE)
gratings provide mid-channel spectral resolutions of $R \sim 3500$ at high diffraction efficiency in each channel.  The
dispersed light is focused by f/2.25 refractive cameras\footnote{Based on the camera design for the DESI
spectrographs.} and recorded by 6k$\times$6k CCDs with 15 $\mu$m pixels and 5-pixel sampling of the fiber diameter.
Spectrographs are mounted in a temperature-controlled housing on the Nasmyth Deck.  The end-to-end instrument
throughput peaks at 60\% and is greater than 30\% at all wavelengths.


\noindent \textbf{Observing modes.}


A total of 1800 150-$\mu$m core diameter fibers are deployed at the curved focal plane.  Fore-optics on the front end
of each fiber demagnify and speed up the beam (from f/15 to f/5) for better coupling to the fiber numerical aperture
and to minimize losses from focal ratio degradation.  The focal plane plate rotates and translates to follow image
positions as the telescope tracks across the sky.  

Sets of 600 fibers feed each of three identical spectrographs (Fig
\ref{fig:layout}).  
%\comment{compare to PEP}.

FOBOS includes observatory level systems for precise instrument
calibration using dome-interior screen illumination, a metrology system
for accurate fiber positioning, and guide cameras for field acquisition
and guiding.  The instrument design envisions future upgrades including
alternate collecting modes that deploy multiple fiber bundles, feeds to
other fiber-based spectrographs at different wavelengths or spectral
resolutions, and the ability to support and benefit from image
corrections with Ground-Layer Adaptive Optics.

\subsection{FOBOS Instrument Design Effort}
\label{sec:design}

FOBOS will complete its current conceptual design phase in fall 2019. Funding from this proposal will support the next
phase of Preliminary Design.  A schedule of milestones and additional information is provided in the Project
Execution Plan (PEP).  Major components of the Preliminary Design effort are described below.

\noindent \textbf{Atmospheric Dispersion Compensator (ADC):} The
opto-mechanical design, tolerancing, lens cell design, motion systems,
and software-control design of the ADC will be completed.  

\noindent \textbf{Focal Plane System:} Mechanical design, including flexure analysis and
the selection of drive mechanisms and potential vendors will be
completed.  This system also defines one of the interfaces to the Keck
II Telescope and must comply with WMKO space envelopes, servicing needs,
and other requirements.  The focal plane system also includes the
guide cameras.

\noindent \textbf{Starbugs fiber positioners:} Starbugs are a
positioning technology developed and deployed by the Australian
Astronomical Observatory (AAO), which has partnered with our team to
generate a conceptual design for use of Starbugs by FOBOS.  Design
requirements for Starbugs in FOBOS are more relaxed than the currently
on-sky TAIPAN instrument thanks to the larger physical plate scale at
Keck.  

\noindent \textbf{Fiber System:} We will complete the optical design and
processing plan for affixing forward optics lenses to each fiber head.  A
micro-lens array solution will be developed for a central,
fixed-position 4.5-arcsec diameter IFU for fast source acquisition. This
work package also includes the stress-relief cable system and fiber
termination hardware and processing.

\noindent \textbf{Spectrographs.} The optical systems and components
(slit, collimator, dichroics, gratings, and camera), an analysis of
acceptable tolerances and performance, their mechanical supports,
software controls, and the overall enclosure will all be advanced
through Preliminary Design.  Detectors, cryostats, read-out electronics
and systems for thermal management will be designed.

% Put the calibration system back in?
% \noindent \textbf{Calibration System.} This package includes design of an interior dome screen and projection system for injecting calibration sources with sufficient spatial uniformity and stability into the instrument.  We will work with the Observatory to develop an integration and controls plan.  No such calibration system currently exists at Keck.

% \noindent \textbf{Auxiliary Systems.} Design of auxiliary systems includes Nasmyth platform interfaces, utilities access, fiber routing and support, thermal control and vibration control systems.

\subsection{Addressing Data Science Challenges and Designing FOBOS Training Sets}
\label{sec:survey}

Our team includes leading experts on data science applications to
astronomy and, specifically, LSST.  We will also use our established
connections to LSST's Informatics and Statistics Science Collaboration
(ISSC) to advertise, recruit, and coordinate efforts to tackle the Data
Science Challenges described in Section \ref{sec:goals}.  Our proposal
request includes two community workshops to motivate progress and discuss
results. At the end of the proposal period, we will publish the results
and developed software packages.

Our data-science challenges require work on simulated
imaging$+$spectroscopic data sets where input physical properties (e.g.,
redshift) can be compared to output recovered values.  Simulated imaging
data (e.g., from LSST and WFIRST) are in-hand, while mock spectroscopy
will be provided by a FOBOS instrument simulator, an initial version of
which has already been developed.  Further advances to be supported by
this proposal include improved error modeling and simulating systematic
effects from detector artifacts, image quality aberrations informed by
the emerging detailed optical design, and variable observing conditions.

The resulting success in addressing each data-science challenge will
define a level of readiness and set requirements on desired FOBOS
training sets, including number of sources, pointings, magnitude limits,
signal-to-noise thresholds, and observing conditions.  Preliminary
observing design and a description of required operational modes to
efficiently observe these training sets will begin with this proposal.
Operational modes will set requirements on target aggregation and
prioritization systems, field acquisition speed, field rotation range,
zenith avoidance zone, reconfiguration time, calibrations, read-out
time, quick-look reduction software and processing rates.  We will
develop integrated program concepts that efficiently combine required
observations.  Detailed survey and execution plans will be completed in
the next phase of this project (MSRI-2).  Roughly 20\% of Keck observing
time is open to the public, and as in previous federally-funded
projects, we fully expect that Senior Personnel at Keck institutions
will be successful in collaborative efforts to secure significant
amounts of additional telescope observing time to enable rapid, public
release of FOBOS training data \citep[e.g.,][]{newman13}.

% The complete photo-$z$ training survey described in \citet{newman15}
% would require 15 independent pointings, each spanning 0.1 deg$^2$ with
% a target density of 6 arcmin$^{-2}$ (8 arcmin$^{-2}$ when including $z
% > 1.5$ galaxies accessible in the UV with Keck-FOBOS), perfectly
% matched to the Keck-FOBOS field-of-view and target density.  With a
% conservative exposure time of 100 hours to reach 75\% redshift
% completeness for 40,000 galaxies with $i_{\rm AB} < 25.3$, the Neman
% survey would require 400 nights.  Challenge \ref{photoz} would reduce
% the required survey duration by a factor of at least four.  Meanwhile
% the extreme depths and flux-limited selection are likely also
% requirements for training sets associated with Challenges \ref{phot},
% \ref{uv}.

% A wider and shallower survey component is envisioned for Challenges
% \ref{lowsnr} and \ref{gaia}.  With 10-minute integrations, a 52
% deg$^2$ Keck-FOBOS sample of environmental diagnostics for 1 million
% galaxies could be carried out in less than 20 nights.  This program
% would sample at $z \sim 1.5$ the same cosmic volume as SDSS.  A
% program of a similar scale would provide training set data for
% inference of stellar parameters in the Milky Way.  These shallow
% programs would be integrated with the deeper components described
% above into a single survey plan.

\subsection{MAISTRO: Target Allocation with Artificial Intelligence}
\label{sec:targeting}

Powered by Starbugs fiber positioners, FOBOS will enable fast, dynamic
reallocation of fibers.  To efficiently determine the best options given
a wide range of possible targets and desired observing outcomes, we will
develop a preliminary design for MAISTRO\footnote{MAISTRO: Modular
Artificial Intelligence System for Target Reallocation and Observing.}
an ``artificial intelligence'' (AI) targeting system that will learn
optimization strategies for assigning targets from a database of
overlapping observing programs with pre-defined priorities.  The AI
package will aggregate data quality using a quick-look reduction
package, science-driven performance metrics, {\it and real-time
assessments of the observing conditions} to make dynamic targeting
recommendations.  For example, if conditions are slightly less than
optimal, MAISTRO would reconfigure Starbugs to brighter objects in a
field or implement a different program prioritization.  MAISTRO will
incorporate updated target lists and priorities from the active observer
and could easily be over-ridden at any time.   Fractions of the full
FOBOS multiplex might also be reserved ``manual targeting'' as required
by the program PI.  

%   - maintains a database with observational progress on individual
%     targets in the survey and
%   - dynamically reallocates fibers based on real-time assessments of
%     the aggregate S/N of each target to meet the specific need of each
%     science case.

% This requires significant design and testing of a combined software
% package and hardware interface.  Specific considerations involve (1)
% fast and robust reduction procedures (cf. MaNGA DOS) that can assess
% the aggregate data and (2) a responsive database with a schema
% optimized for real-time decision making to select targets for
% (re)acquisition while accounting for collision limitations.  Provided
% enough design effort, this lends itself to a machine-learning
% application.







\subsection{Management}


% FOBOS has a detailed Project Management Plan (PEP) which follows guidelines outlined in Section 3 of the NSF Major Facilities Guide (NSF 19-XX Dec 2018).  The PEP is outlined at Level 1 for this pre-proposal.  Level one PEP outlines is as follows; 1. Introduction, 2.  Organization, 3.  Design and Development, 4.  Contraction project definition, 5.  Staffing, 6.  Risk and Opportunity Management, 6.  Systems Engineering, 7.  Configuration and Controls, 8.  Acquisitions, 9.  Project Management Controls, 10.  Site and Environment, 11.  Cyber-infrastructure, 12.  Environmental Safety and Health Plans, 13.  Reviews and Reporting, 14.  Commissioning, 15.  Project Closeout plan

% Project management controls for this and future project phases is accomplished by low level task lists which are used to develop resource loaded schedules for determining both cost and duration.  This ‘bottom up’ approach allows for a detailed accounting of project cost and cash flow.  It allows for in process tracking trough earned value reporting, earned value will be calculated a minimum quarterly or when deemed necessary by the Project Manager.  Each sub-award institution is responsible for maintaining their own schedule and budget as well as reported earned value to the Project Manager.  A detailed Work Breakdown and Product Breakdown structure is in place to insure clear understand scope of delivered work and to insure well documented interfaces between systems.

% Contingency for this proposal, funding of preliminary design, is set at 20\%.  As the project develops the contingency on future work and equipment will be set using a combination of basis of estimate and associated risk.  The process for contingency estimation is included in the detailed PEP.

\subsection{Publicly Available Automated Data Products}
\label{sec:DAP}

While the FOBOS data simulator is required for our data-science challenges, it also forms the basis of a delivered data
reduction pipeline (DRP) for this instrument.  This software will provide both the quick reduction assessments needed
for dynamic targeting, as well as full reductions for scientific analysis.  In the proposal period, we will also develop a preliminary design for a data analysis
pipeline (DAP).  Unique among Keck instruments, the FOBOS DAP will take advantage of the fixed spectral format and common target classes to provide high-level data products, including
Doppler shift, emission-line strengths, and template continuum fits (cf.,
Westfall et al.; SDSS-IV MaNGA DAP).  The DAP will also produce results from relevant machine-learning applications (e.g., redshifts at low-S/N).

Raw data, reduced spectra, and high-level DAP science products will be
publicly delivered via user-friendly platforms built on the Keck Observatory Archive.  After associated
proprietary periods, data will be served for {\it all} FOBOS observations, creating a rich legacy data set for the astronomical community.  Both program PIs
and the larger community will be encouraged to develop the DRP and DAP
to meet the needs of specific science applications.  These software
packages will be open source and publicly served (e.g., using GitHub).

\section{Broader Impacts}
\label{sec:bi}

\subsection{Akamai: Training the next generation of Hawaiian STEM
professionals} Led by the Institute for Scientist and Engineer Educators
(ISEE) at UCSC, the Akamai Internship Program is aimed at advancing
college students from Hawai'i into the STEM workforce.  Almost 400
students have participated to date, of which 24\% are Native Hawaiian
and 38\% are women. A longitudinal study of Akamai outcomes indicated
that 87\% were still in STEM, either in the workforce or continuing STEM
studies \citep{asee_peer_31030}.  ISEE and the Akamai program already have deep connections to WMKO; 45 interns have
worked on projects related to instrument
development and observatory operations over the past 15 years.  Our
funding request includes support for two Akamai interns.\footnote{
%
At UCO, an Akamai intern during Summer 2018 helped build a fiber
test-bench at UCSC.}
%
One intern will develop aspects of the FOBOS instrument simulator and use this simulator to develop performance metrics
for the Preliminary Design.  The second will build machine-learning tools for Data-Science Challenge \ref{lowsnr},
specifically for obtaining spectroscopic redshifts at low S/N.


\subsection{Investing in future educators} Also via the ISEE, we will
support three graduate students to participate in the Professional
Development Program (PDP) that build teaching skills through
collaborative design of an inquiry activity.  The PDP team conceives,
develops, and tests the activity which culminates in a lab exercise
run with undergraduates. The program
emphasizes inclusive and equitable learning environments.  Specifically, our team of graduate
students will develop a lab unit related to FOBOS instrument development aimed at
incoming community college transfer students enrolled in UCSC's
highly-successful Lamat Program.  In addition to enriching graduate
student training, these efforts will positively impact 25
undergraduates from California community colleges, a large fraction
from underrepresented minority groups.

\subsection{Student Training} UCSC's {\bf Astro 9} course introduces scientific research methods to 1st-year students
by engaging small student teams on actual research projects supervised by graduate students,
postdocs, and staff.  The {\bf Science Internship Program} (SIP) creates a similar environment for high-school students
 over a 10 week summer program.  We will design projects for both programs focused on simulating data sets and
 introducing machine learning concepts used in our Data Science Challenges.  Both PI Bundy and co-PI Westfall have
 served as research mentors in these programs.

\newpage

\setcounter{page}{1}
\bibliographystyle{nsf}
\bibliography{../../references}

\end{document}


