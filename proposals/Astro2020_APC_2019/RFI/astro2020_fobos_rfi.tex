\documentclass[oneside,11pt]{amsart}

%\usepackage{a4wide}
%\usepackage{epsfig}
%\usepackage{psfig}
\usepackage{graphicx}
\usepackage{natbib,latexsym,url,enumitem,pdfpages,multicol}
\usepackage{color}
\usepackage{wrapfig}
\usepackage[belowskip=-10pt,aboveskip=0pt]{caption}
\usepackage{threeparttable}

\captionsetup{
    justification=justified,
    margin=0pt,
    font=small}

%%%%%%%%%%%%%%%%%%%%%%%%%%%%%%%%%%%%%%%%%%%%%%%%%%%%%%%%%%%%%%%%%%%%%%%%
% Allow for the okina; thanks to:
% https://tex.stackexchange.com/questions/424535/how-to-type-a-proper-hawai%CA%BBian-%CA%BBokina

\usepackage[utf8]{inputenc}
\usepackage{newunicodechar}
%\usepackage{libertine}

\DeclareRobustCommand{\okina}{%
  \raisebox{\dimexpr\fontcharht\font`A-\height}{%
    \scalebox{0.8}{`}%
  }%
}
\newunicodechar{ʻ}{\okina}
\newcommand{\hawaii}{Hawaiʻi}
%%%%%%%%%%%%%%%%%%%%%%%%%%%%%%%%%%%%%%%%%%%%%%%%%%%%%%%%%%%%%%%%%%%%%%%%

\newcommand{\arcmin}{\mbox{$^{\prime}$}}
\newcommand{\arcsec}{\mbox{$^{\prime\prime}$}}
\newcommand{\gt}{$>$}

% Some fancy commenting
\definecolor{todo}{RGB}{200,0,0}
\newcommand{\comment}[2][todo]{{\color{#1}[[{\bf #2}]]}}

% Challenge counter
\newcounter{chalno}
\newcommand{\chal}[1]{\refstepcounter{chalno}\label{#1}}

% User commands
\input{journaldefs}

\DeclareRobustCommand{\gtrsim}{%
\mathrel{\hskip-.5em\begin{array}{c}>\\[-8pt]\sim\end{array}\hskip-.5em}}
\DeclareRobustCommand{\lesssim}{%
\mathrel{\hskip-.5em\begin{array}{c}<\\[-8pt]\sim\end{array}\hskip-.5em}}


\pretolerance=10000
\textwidth=6.4in
\textheight=8.95in
\voffset = 0.in
%\voffset = -0.3in  % For my printer
\topmargin=0.0in
\headheight=0.00in
\hoffset = 0.0in
%\hoffset = 0.33in  %  For my printer
\headsep=0.00in
\oddsidemargin=0in
\evensidemargin=0in
\parindent=2em
\parskip=0.2ex
 
\renewcommand{\baselinestretch}{1.03}

\special{papersize=8.5in,11in}

\newcommand{\markus}{\textcolor{green}}

\setlength{\parskip}{0.6 ex plus 0.4ex minus 0.2ex} \flushbottom
\pagestyle{plain} 

\begin{document}
% \thispagestyle{empty}

\pagenumbering{arabic}

\vspace*{-1.5cm}

\centerline{\textsf {\Large FOBOS: A Next-Generation Spectroscopic Facility}}
\centerline{\textsf {\large Response to Astro2020 Request for Information}}

\setcounter{page}{1}

\section*{Executive Summary}

% - Summarize your science objectives and your technical implementation at
%   a high level.

\noindent{\bf Science objectives:} FOBOS, the Fiber-Optic Broadband
Optical Spectrograph, is a facility-class, general-purpose
spectrograph for the 10m Keck II telescope. It emphasizes UV
sensitivity (up to the atmospheric limit), high multiplex, multiple
focal-plane sampling formats, and near Poissonian-level performance
over extremely long ($\sim$100 hr) integration times. Its flexible
position system enables advanced techniques for merging multiple
observing programs, including a constant fraction of fibers (5-10\%)
allocated to the full US community amounting to $\sim$100,000
fiber-hours per year. This establishes FOBOS's uniqueness, both in
term of functionality and access, among the suite of spectrographs
coming online for 8-10m class telescopes in the next decade. FOBOS is
specifically built to enable the deep spectroscopic follow-up of
upcoming large-scale imaging surveys (LSST, WFIRST, and Euclid)
identified as a significant need of the astronomy community in the US
and beyond. For example, capitalizing on the combined $\approx$\$4B
the US is investing in these imaging surveys, FOBOS can increase
LSST's dark-energy figure-of-merit by 40\% \citet{newman15}.

\smallskip

\noindent{\bf Technical implementation:} FOBOS is a fiber-fed,
fixed-format spectrograph. Its primary systems include (1) a
compensating lateral atmospheric dispersion corrector (CLADC), (2) a
robotic fiber-positioning system based on AAO's Starbug technology,
(3) micro-assembly fore-optics required to optimally couple the Keck
II input beam to the fiber feed, (4) 10m-long fiber trains that
provide at least three focal-plane sampling options (single fibers,
multiple small IFUs, and a single large-format IFU), and (5) a bank
of three 4-channel spectrographs that provide a simultanous spectral
range of 0.31-1 $\mu$m at a resolution of $R\sim3500$ across the full
band.

\smallskip

% - Summarize the technology maturity of your implementation, listing the
%   demonstrated technologies and the technologies requiring development.

\noindent{\bf Technological maturity:} FOBOS is nearing the end of
its conceptual design phase, and will begin its preliminary design
phase in Oct 2020, contingent on funding. Components of this document
represent significant improvements in FOBOS's design since the
completion of our Astro2020 submission. Although requiring
significant design work, FOBOS's backend systems are built on
well-established technologies: FOBOS's spectrographs and fiber train
inherit significantly from SDSS, MaNGA, and DESI designs. FOBOS's
front-end leverages the more than 15 years of development of the
Starbugs position system, which have been deployed and are in the
final commissioning stages for the TAIPAN instrument on the AAT.
Further development of the Starbugs technology is needed for use in
FOBOS; however this represents a 2nd-generation implementation of the
technology and will leverage simultaneous development of the
GMT/MANIFEST instrument. Further development of the micro-assembly
fore-optics is also critical; prototypes are currently being
developed with potential industry partners and tested at UCO/UCB.

\smallskip

% - Summarize areas where the data to support this RFI are not currently
%   available.

\noindent{\bf Limitations of this RFI:} Given this early stage in
development, particular project components with limited extant
data/information to address this RFI are as follows:

\newpage

\section{Science}

% - Briefly describe the scientific objectives and the most important
%   measurements required to fulfill these objectives. Feel free to refer
%   to science white papers or references from the literature.

FOBOS, the Fiber-Optic Broadband Optical Spectrograph, is a
facility-class, general-purpose spectrograph for the 10m Keck II
telescope. Although its scientific use is expected to be very broad,
FOBOS meets an explicit need of the astronomy community in the era of
deep imaging surveys, like LSST, WFIRST, and Euclid, allowing for
highly efficient follow-up spectroscopy of faint sources using one of
the world's largest telescope. The design of the instrument reflects
this need by emphasizing high target densities, multiple focal-plane
sampling formats, blue sensitivity, and a stabilized and
well-calibrated instrument performance. To help establish these
requirements, we have outlined three ``design-reference'' key science
programs that facilitate the definition of FOBOS's instrument
requirements. These key programs are:

\smallskip

\noindent{\bf (a) Dark Energy}: Distance measurements for the
billions of sources in deep-imaging campaigns, like LSST, WFIRST, and
Euclid, will hinge on the use of photometric redshifts (photo-$z$s).
FOBOS observations that target $\sim$15k sources with $24 > i_{\rm
AB} > 25.3$ in specific regions of galaxy color space \citep[see][]{masters15, masters19} can dramatically improve the photo-$z$
accuracy applied to the full LSST sample, equivalent to an {\em
increase of LSST's dark energy figure-of-merit by 40\%} \citep{newman15}. The {\it key instrument requirements} derived
from this program are (1) the blue wavelength coverage (down to the
atmospheric limit) to eliminate the usual "redshift desert", (2)
long-term stability of the instrument to allow for
near-Poisson-limited data reduction for ultra-deep, $\sim$100-hr
integrations, and (3) high multiplex to meet the target-density needs
of sources to $i_{\rm AB} > 25.3$ ($\sim$40 per arcmin$^2$).

\smallskip

\noindent{\bf (b) The baryonic ecosystem of galaxies at cosmic high
noon}: Mapping the crucial link between galaxies at $z\sim$2--3 and
the extended gas reservoirs, diffuse halos, and streaming filaments
that dominate the mass and regulate galaxy evolution in these
environments requires an instrument like FOBOS, even in the era of
JWST and ELTs. Its deep sensitivity and high sampling density enables
comprehensive tomographic reconstruction of the intergalactic medium
(IGM) across the largest cosmic structures in a single pointing
($\sim$10 transverse Mpc at $z \sim 2.5$). Its blue sensitivity
probes Ly-$\alpha$ across the complete formation epoch ($z =
1.5$--3.5) and opens access to high-ionization transitions that
reveal diffuse gas \emph{in emission}, such as O VI (1032 \AA).
Finally, its ability to combine single-fiber and multiplexed IFU
observations allows us to map the density and dynamical state of
diffuse gas at all relevant scales from the IGM to the circumgalactic
medium (CGM). The {\it key instrument requirements} derived from this
program parallel those from our ``Cosmology'' program, and
additionally require an integral-field mode that provides
two-dimensional kinematic data.

\smallskip

\noindent{\bf (c) The chemodynamical history of M31}: Compared to
existing and forthcoming spectrographs (e.g., PFS, MSE), FOBOS's high
sampling density and IFU modes offer unique capabilities for
efficiently mapping the high-source-density M31 disk. Building on
previous imaging \citep[PHAT][]{phat} and spectroscopy \citep[SPLASH][]{splash} FOBOS will provide high-S/N spectra of $\sim$100k M31
disk stars that connect [Fe/H] and [$\alpha$/Fe] patterns to the
underlying dynamics. FOBOS will use integral-field observations of
$\sim$150 young stellar clusters that yield their present-day mass
function. Additionally, a few larg-scale star-formation complexes are
observed with FOBOS's monolithic IFU (FOV $\sim$ 700 arcmin$^2$)
allow for comparison of the local star-formation activity with the
Milky Way. In general, integral-field observations of the M31 disk
enable unique background-subtraction techniques that more accurately
remove the underlying diffuse M31 light, unavailable to single-fiber
observations. Combining these FOBOS observations with integral-field
data from the SDSS-V Local Volume Mapper and PFS/MSE surveys of halo
structure, a complete picture of the Andromeda system's formation
history will address key questions about disk evolution, dwarf
galaxies, and dark matter substructure with a level of statistical
power that has so far been limited to the Milky Way. The {\it key
instrument requirements} derived from this program parallel our
previous programs and additionally motivate the monolithic
integral-field unit.

~\comment{discuss community access here?}

Beyond the scope and requirements of these ``design reference''
programs, {\bf FOBOS enables a broad range of observations}, e.g.:
Milky Way and M31 halo stars and substructure; the Milky Way bulge;
globular clusters; variable stars from cadenced LSST imaging; dwarf
galaxies; rapid time-domain followup with an always-ready IFU;
structure of Coma and Virgo galaxies with IFUs and using globular
clusters and PNe as tracers; large IFU samples at $z \sim 1$ of 2D
emission-line kinematics; galactic winds; radial stellar-population
trends from stacked spectra at $z \sim 1$; environmental group
identification at $z \sim 1$--2; galaxy cluster and proto-cluster
followup; QSO light echos in the IGM; and redshift calibration of LBG
samples at $z = 1.5$--5 for CMB lensing cross-correlation.

Finally, given the wealth of photometric data and the continued
increase in sophistication and broad application of state-of-the-art
machine-learning techniques, we expect a common theme of FOBOS
observations to be the build-up of optimized training samples. We
have already seen the power of the combination of large samples of
photometric or low-spectral-resolution data with higher resolution
training sets in inferring higher-order properties. Examples include
stellar parameters (age, metallicity, $\log g$, $T_{\rm eff}$) and
distance (e.g., Ting et al. 2018a,b) and emission-line fluxes in
galaxy spectra (e.g., Capak).

% - Of the objectives, which are the most demanding? Why?
% - Present the highest-level technical requirements (e.g. spatial and
%   spectral resolution, sensitivity, timing accuracy) and their relation
%   to the science objectives.

\subsection{Top-level Requirements}

Table \ref{tab:reqs} presents the top-level requirements for FOBOS as
they stand at the time of this writting; however, we caution that
these are still under development.

\begin{table}[h!]
\centering
\begin{threeparttable}
\caption{FOBOS Top-Level Requirements}
\footnotesize
\begin{tabular}{| l | c | p{2.5cm} | p{1.5cm} | p{4cm} |}
\hline
{\bf Description} & {\bf ID} & {\bf Requirement} & {\bf Goal} & {\bf Justification} \\
\hline
\hline
Field of View & TOP.REQ.A01 & $D=17\arcmin$ & $D=20\arcmin$ & Maximum FoV matches full Keck-Nasmyth FoV \\
\hline
MOS-fiber multiplex & TOP.REQ.A02 & 1800 &  & Achieve density of 6 arcmin$^2$ for $20\arcmin$ FoV \\
\hline
MOS-fiber aperture & TOP.REQ.A03 & $0.7\arcsec < D < 1.3\arcsec$ &  & Optimize S/N of extracted spectra given WMKO seeing distribution \\
\hline
Calibration mini-IFUs & TOP.REQ.A10 & 12 7-fiber bundles & & Flux calibration fidelity; four per spectrograph (Yan et al. 2016) \\
\hline
MOS-IFU size & TOP.REQ.A04 & $\gtrsim$5\arcsec &  & Cover scales of $\gtrsim$20 kpc at $z=2-2.5$ \\
\hline
MOS-IFU sampling & TOP.REQ.A05 & $\lesssim$0.8\arcsec & & Optimize extended-source spectroscopy \\
\hline
MOS-IFU multiplex & TOP.REQ.A06 & 45 & & 37 fibers per IFU, 15 IFUs per spectrograph \\
\hline
Monolithic IFU FoV & TOP.REQ.A11 & $D\sim$30\arcsec & & Maximum size for full fiber complement in a single IFU \\
Wavelength coverage & TOP.REQ.A07 & 0.31 - 1 $\mu$m & & Eliminate redshift desert \\
\hline
Spectral Resolution & TOP.REQ.A08 & $R = 3500$ & & Subtraction of night-sky lines in the red \\
\hline
Throughput & TOP.REQ.A09 & $\gtrsim$30\% over 95\% of the bandpass & & Competitive with other upcoming instruments \\
\hline
\end{tabular}
\begin{tablenotes}
\item MOS-IFU requirements above are optimized for the CGM science
case; however, we expect additional IFU modes to be explored,
including those that critically sample the median Keck seeing
(0.6\arcsec) and those that take advantage of a GLAO mode when it is
available at Keck.
\end{tablenotes} 
\label{tab:reqs}
\end{threeparttable}
\end{table}

The most demanding aspects of our design-reference science programs
on FOBOS's performance and instrument requirements are in its ability
to reach near-Poisson level performance for very long integrations
($\gtrsim$100-hr). Analysis of existing data from the MaNGA/SDSS-IV
survey, a survey and instrument not meant to enable such long
exposures, demonstrates a sky-subtraction fidelity of $\sim$0.2\%
(Childress et al. 2017, MNRAS, 472, 273; Bundy et al., in prep; Gu et
al.). Early commissioning data for DESI shows similar performance. A
key aspect of our design and instrument simulation work during the
instrument design phases will be to (re)assess the instrument,
calibration, and data-reduction-software systems with respect to
their expected performance over very long integration times. This
work will continue to inform the relevant system requirements well
into final design. Deviations from Poisson performance would lead to
longer integration times for our Cosmology and CGM programs; however,
the performance of current fiber-based instruments is such that we do
not expect these deviations to be significant enough to render these
programs infeasible.

% - For each performance requirement identified, describe as
%   quantitatively as possible the sensitivity of the science objectives
%   to achieve the requirement.  If you fail to meet a key requirement,
%   what would be the impact on achieving the science objectives?

\section{Enabling Technology}
\label{sec:tech}

% - Please provide information describing new enabling technologies
%   required for activity success.

There are no fundamentally new technologies needed to enable the
FOBOS concept to be realized. However, there are a few key
technologies that require new design innovations or modifications of
existing designs for there specific use in FOBOS.

% driven by low-risk

% LRS2 microlens arrays, full re-imaging system (very hard alignment problem, but lots of real-estate; PFS has microlens on the fiber, but bonded)  Need for airgap, Starbugs packaging

FOBOS's conceptual design includes fused-silica etched (FSE) gratings
in each channel of our 4-channel spectrographs. To date, FSE gratings
have had limited use in astronomical instruments; however, FSE
grating advancement has come quickly on the heels of development of
e-beam lithography techniques, which allow shaping of grating grooves
to nanometer precision. Custom designed gratings promise high
efficiency over our entire wavelength range. In particular, their use
in FOBOS overcomes a lack of volume-phase holographic (VPH) gratings
with good transmission at wavelengths shorter than 350 nm.
Prototyping and testing of these diffractive optics will be ongoing
during our preliminary design phase. If necessary, our design can be
rescoped to use VPH gratings with appropriate compensations to the
rest of the design given their limitations.

Additional key design elements are:
%
\begin{enumerate}
\item Implementation of a compensating lateral atmospheric dispersion
corrector (CLADC) that includes a fold mirror between its 2nd and 3rd
elements. The mirror folds the incoming beam such that the face of
the final element of the CLADC is perpendicular to the gravity vector
and serves as the drive surface for the Starbugs positioning system
(see below). This orientation dramatically decreases the risk of
Starbug adhesion failures on the drive surface.
\item Development and process testing of micro-assembly foreoptics
composed of single lenses for single-fiber apertures and microlens
arrays for integral-field apertures.
\item Integration of both single-aperture and multi-aperture arrays
assemblies in the Starbugs piezo assembly.
\item Roughly an order-of-magnitude increase in the number of Starbugs
simultaneously used in a positioning system.
\end{enumerate}

% - Please indicate any non-US technology required for activity success
%   and what back up plans would be required if only US participation
%   occurred. 

\subsection{Non-US Technologies}
\label{sec:starbugs}

The fused-silica etched gratings in our current design are a
proprietary technology of the Fraunhofer Institute for Applied Optics
and Precision Engineering (IOF) in Germany. The IOF produce the most
efficient FSE gratings in the world, as needed for the unique
sensitivity requirements for FOBOS and particularly in its UV channel
(see above).

The FOBOS robotic focal-plane positioning system uses the Starbugs
technology, developed exclusively by Australian Astronomical Optics
at Macquarie University in Sydney, Australia. No similar technology
exists from a US vendor or otherwise. The Starbugs technology is very
attractive for FOBOS for three reasons: (1) It has the potential to
allow for multiple focal-plane modules specifically designed with an
exchange procedure that can be performed during daytime telescope
operations. (2) Similarly, the system allows for upgrade paths that
add new focal-plane sampling modes/modules that either interface with
the currently proposed FOBOS spectrographs or new spectrographs that
provide expanded capabilities (e.g., higher spectral resolution). (3)
Starbugs allow for the very large patrol radii needed for the
mini-IFU modes, in particular.

Starbugs are currently deployed by the TAIPAN instrument on the
Anglo-Australian Telescope (AAT). TAIPAN's commissioning phase has
proven critical to the refinement of the Starbugs design, and the
instrument is nearly ready to enter normal operations. AAO is deeply
invested in the future of the Starbugs technology, both via our
interest in their use in FOBOS, but also via their development of
MANIFEST, a front-end for the first-light spectrographs built for the
Giant Magellan Telescope.

In the event that the Starbugs technology becomes untenable for
implementation in FOBOS, we will explore zonal position systems
similar to those used by DESI, PFS, and SDSS-V. This may require a
descope of the number of focal-plane sampling modes available or
exploration of other technologies that may enable these modes, such
as highly efficient fiber switching.

% - For any technologies that have not been demonstrated by sub-scale or
%   full-scale models as of this request, please describe the rationale
%   for your technical maturity assessment, including the description of
%   analysis or hardware development activities to date, and its
%   associated technology maturation plan.
% TODO: Come back to this.

% - Describe the aspect of the enabling technology that is critical to the
%   concept’s success, and the sensitivity of mission performance if the
%   technology is not realized or is only partially realized.
% TODO: Assess the sensitivity of FOBOS to each of the technologies listed above?

% - Provide cost and schedule assumptions by year for all development
%   activities, and the efforts that allow the technology to be ready when
%   required, as well as an outline of readiness tests to confirm
%   technical readiness level.
% TODO: Just refer to the cost section?

\section{Telescope}

FOBOS is an instrument planned for the existing Keck II telescope.
Installation, integration, and commissioning of the instrument are
minimally described in our work breakdown structure (see Section
\ref{sec:cost}), with specific details to be worked out in
consultation with our Keck partners during our upcoming preliminary
design phase.
 
% These details are irrelevant to FOBOS:
% - Provide an overview description of the characteristics and
%   requirements of the optical telescope(s), antenna(s), or collector(s)
%   highlighting key capabilities and any residual technology risks. 
% - Provide diagrams or drawings showing the observatory or antenna array
%   with the instruments and other components labeled and a descriptive
%   caption.
% - Please provide any available review packages (e.g. Conceptual Design
%   Review, Preliminary Design Review) that describe the scope of
%   technical design and implementation.
% - Please describe any hardware/software with significant heritage.
% - Please fill out the table below regarding the primary scientific
%   equipment (e.g., Telescope or Antenna Array). Expand, contract, or
%   modify this table as necessary and applicable.
% - Identify and describe the three components of lowest technical
%   maturity, and explain how and when these components will be
%   demonstrated in hardware.
% - What are the three greatest risks to cost, schedule, and performance?
% - Describe any aspect of the design or implementation that may require
%   non-US participation.

\section{Instrumentation}

\begin{figure}[h!]
\vskip -0.1in
\includegraphics[width=\textwidth]{FOBOS_inst_2019-10-28.pdf}
\caption{\small {\it Left}: Rendering of FOBOS instrument systems
deployed at the Keck II Nasmyth port. {\it Right}: Optical layout of
one of the three four-armed FOBOS spectrographs.}
\label{fig:layout}
\end{figure}

% - Describe the proposed science instrumentation, and briefly state the
%   rationale for selection. Discuss the capabilities of each instrument
%   (Inst #1, Inst #2 etc.) and how the instruments are used together.
%   Indicate whether cryogens or other cooling are required. 

% - Briefly describe any concept, feasibility, or definition studies
%   already performed and please provide any available Review Packages
%   (e.g., Conceptual Design Review, Preliminary Design Review) that
%   describe the instrument and its design and implementation.

The current conceptual design of FOBOS inherited much from the
concept of a fiber-based optical spectrograph for the Thirty-Meter
Telescope in a trade/down-selection study for its Wide-Field Optical
Spectrograph (WFOS). FOBOS is still in its conceptual design phase,
and therefore no CDR package has yet been produced or included in
response to this RFI.

% - What are the three primary technical issues or risks?
% - Describe any instrumentation that may require non-US participation. 

The current conceptual designs for all FOBOS sub-systems are briefly
described in the following subsections, with its instrument
specifications summarized in Table \ref{tab:instrument}. We also
briefly comment on the technical maturity and heritage of the
components within each sub-system. Our risk register assigns the
{\bf three highest risk levels} to:
%
\begin{enumerate}
\item the performance (uniformity and stability) of our calibration
system and calibration procedures,
\item the Starbugs positioning system (the only system provided by a
non-US partner; see Section \ref{sec:starbugs}), specifically
adhesion and fine-position control, and
\item the micro-assembly fore-optics, specifically the manufacturing,
assembly, and alignment process control.
\end{enumerate}

\begin{wrapfigure}{r}{0.35\textwidth}
\small
\includegraphics[width=0.35\textwidth]{starbugs_v1.jpg}
\caption{Starbugs mounted at the focal plane of the AAT. Light
enters from the image bottom, where the red or gray
``slipper'' of each Starbug contacts a field plate and is adhered to
it via a leaky vacuum. Each Starbug houses
a single-fiber aperture (white). For FOBOS, the contact surface will be
an optical element of the CLADC, always perpendicular to the
gravity vector, and the Starbugs will be adapted to hold either a
single-aperture microlens assembly or an integral-field microlens
array.}
\label{fig:org}
\end{wrapfigure}

\subsection{Focal-Plane Module}

Light from the Keck II telescope Nasmyth port (Fig \ref{fig:layout};
left) passes through the first two 946mm-diameter lenses of a 3-lens
CLADC before encountering a 45$^\circ$ mirror that folds the beam
vertically upwards. The third lens of the CLADC is horizontal with
the surface always perpendicular to the gravity vector. Positioned at
the focal plane of the telescope$+$CLADC optics, this also serves as
the mounting surface for downward-pointing Starbugs. The risk of
Starbug adhesion loss and focal-plane coupling failure is
significantly reduced by using a horizontal mounting surface
($20\arcmin$ diameter). The mounting surface rotates to match the
field rotation of the telescope as it tracks the sky. Each Starbug
patrols zones of up to several arcminutes and can be placed as close
as 10\arcsec. Three back-illuminated fibers embedded in the housing
of each Starbug are combined with a fast, imaging metrology camera
that enables reconfiguration times of as little as 2 minutes. See
Section \ref{sec:starbugs} for a discussion of their heritage,
maturity, and advantages.

\subsection{Focal-Plane Sampling \& Robotic Positioning System}

Microlens fore-optics coupled to each fiber demagnify and speed up
the telescope beam from $f/15$ to $f/5$. This provides better
coupling of the fiber to the telescope beam --- minimizing losses
from focal-ratio degradation --- and provides an $0.8\arcsec$ on-sky
diameter of each fiber. Integral-field observations are accomplished
by coupling large-fill-factor microlens arrays to fiber bundles. Team
members have experience with microlens coupling to fibers
(SDSS-III/APOGEE) and we are currently prototyping microlens optics
and optomechanical assemblies for lab and on-sky testing. Lab testing
is being performed at both UCO and UCB/SSL, and we are exploring
partnerships for the at-scale manufacturing of these components.

Separate Starbug designs will be produced to house either the
single-fiber or integral-field assemblies. These Starbugs will be
incorporated into separate modules that are stored outside the focal
plane and can be exchanged with the module attached to the focal
plane via a day-time procedure. This allows FOBOS to switch between
different fiber suites: 1) individual fibers, 600 per spectrograph;
2) A set of 15 multiplexed fiber-bundle IFUs (per spectrograph), each
composed of 37 fibers and spanning up to 5.6\arcsec; 3) A
30\arcsec-wide monolithic IFU\footnote{This mode would be unique at
Keck. Compared to KCWI+KCRM, FOBOS's monolithic IFU will emphasize
FoV over spatial resolution and blue sensitivity. KCWI+KCRM covers
0.36--1.0 $\mu$m and to achieve $R \sim 3600$ over the full bandpass,
KCWI+KCRM reduces its FoV to 8.4 $\times$ 20.4 arcseconds, an area
5$\times$ smaller than the FOBOS monolithic IFU.} (1641 fibers). In
all cases, $\sim$10\% of individual fibers are reserved for sky
sampling and 5--10 mini-bundles (7 fibers) observe stars in the field
for flux calibration and PSF monitoring.

Current designs for both the mini-IFUs and the monolithic IFU flow
from our science requirements; however, we are still exploring other
possible modes, enabled by the flexibility of the Starbugs
positioning system and focal-plane module design. In particular, our
design will accept and take advantage of ground-layer adaptive optics
(GLAO) corrections from an anticipated GLAO system at Keck II. GLAO
improves depth, enables crowded source targeting, and opens new
science territory through spatially-resolved galaxies beyond
$z\sim0.5$. Particularly for the latter, we would expect to deploy an
IFU mode that critically sampled the improved GLAO PSF.

\subsection{Fiber-feed \& Spectrographs}

Three spectrographs are mounted on the Nasmyth deck and connected to
the adjacent focal-plane module (Fig \ref{fig:layout}; right) via a a
short ($< 10$m) fiber run (not pictured) in order to preserve UV
throughput (cf., the much longer fiber runs for prime-focus
instruments, such as Subaru-PFS and DESI). Stress-relief features in
the cabling reduce focal-ratio degradation. Our team has significant
experience with fiber-fed spectrographs, drawing from SDSS and DESI
heritage.

Each of the three spectrograph uses dichroics to divide the 140 mm
diameter collimated beam into four wavelength channels with combined,
instantaneous coverage from 0.31--1 $\mu$m. High-efficiency
fused-silica etched (FSE) gratings provide mid-channel spectral
resolutions of $R \sim 3500$ for all channels. The spectrographs use
$f/2.25$ refractive cameras and 6k$\times$6k, 15 $\mu$m-pixels CCD
detectors. The camera platescale is such that the images of the
150$\mu$m fiber cores are sampled by 5 pixels in all channels. Our
spectrograph design draws significantly from the design of the DESI
spectographs.

The spectrographs are mounted in a permanent temperature-controlled
housing, providing a stable environmental temperature ($\pm$1C). Heat
rejection of electronics components in the dome is done through a
glycol cooling loop. Cyrogenic cooling of the science detectors is
provided by liquid N$_2$. The estimated end-to-end instrument
throughput peaks at 60\% and is greater than 30\% at all wavelengths.

\subsection{Calibration System}

FOBOS has a comprehensive strategy for precise calibrations, drawing
on significant heritage from purpose-built calibration systems for
SDSS, DESI, and PFS. FOBOS's calibration system is critical to its
ability to provide near-Poisson performance for very long
integrations. Specific concerns involve the stability of the fiber
input and output beams both between fibers and as a function of time
for a given fiber.

Our current concept for the calibration system and calibrations
procedures are as follows. Afternoon flat-field and arc-line
exposures will use a carefully illuminated interior dome screen,
acquired specifically for FOBOS calibrations but also as a general
good for the Observatory. Nighttime calibrations involve re-orienting
the fold mirror under the focal plane to accept light from a nearby
light source and optical assembly that mimics the telescope pupil.
Afternoon and relative fiber-to-fiber measurements from nighttime
calibrations will be combined to model the instrument$+$sky response
during observations.

\begin{table}[h!]
\centering
\caption{FOBOS Instrument Specifications}
\footnotesize
\begin{tabular}{| l | r | r |}
\hline
{\bf Item} & {\bf Value} & {\bf Units} \\
\hline
\hline
Type of instrument      & Spectrograph & \\\hline
Number of spectrographs & 3 & \\\hline
Multiplex & 1800 & single fibers \\
                        & 45 & 37-fiber IFU \\
                        & 1 & 1641-fiber IFU \\\hline
Field of view: Patrol   & 315 & arcmin$^2$ \\\hline
Field of view: Fiber    & 0.5 & arcsec$^2$ \\\hline
Field of view: Mini-IFU & 25 & arcsec$^2$ \\\hline
Field of view: Large-IFU & 705 & arcsec$^2$ \\\hline
Spectral range          & 0.31-1 & $\mu$m \\\hline
Spectral resolution     & 3500 & $R=\lambda/\delta\lambda$ \\\hline
Number of detectors     & 12 & 4 per spectrograph \\\hline
Detector size           & 6k $\times$ 6k & \\\hline
Thermal requirements    & & \\\hline
Size: Spectograph bank  & & m $\times$ m $\times$ m \\\hline
Size: Focal-plane module & & m $\times$ m $\times$ m \\\hline
Data volume (Section \ref{sec:ops})           & $<$100 & Gb \\\hline
Development Schedule    & 100 & months \\\hline
\end{tabular}
\label{tab:instrument}
\end{table}

% - Indicate the technical maturity level of the major elements and the
%   specific instrument maturity of the proposed instrumentation (for each
%   specific Inst #1, Inst#2 etc.), along with the rationale for the
%   assessment (i.e. examples of heritage, existence of breadboards,
%   prototypes, mass/volume and power comparisons to existing units, etc.
%   and any identifications of major long lead items). 

% - Describe the heritage of the instruments and associated sub-systems.
%   Indicate items that are to be developed, as well as any existing
%   hardware or design heritage.

% - Please fill out the table below regarding each instrument (if
%   applicable). Copy as needed for all instruments. Expand this table as
%   necessary and applicable.

\section{Facilities}

% - If a site is not yet selected for the project, describe the
%   anticipated approach to conducting site studies, obtaining site
%   permissions, and executing environmental impact studies.
% - Please provide any available site plans. If such plans already address
%   items 2-5 below, these items do not need to be addressed separately.
% - Describe the site and its location, including size, altitude, access,
%   number of buildings, size of building(s) footprint and volume,
%   existing infrastructure, power, internet, environmental considerations
%   and logistics (proximity to major airport, housing and support for
%   construction crews and facility staff etc.).
% - Identify which facilities will be new and which facilities may be
%   pre-existing. Describe any existing facilities and their estimated
%   remaining useful life. Describe any upgrades to existing facilities
%   that will be undertaken. Describe any anticipated shared use of site
%   facilities between the concept being proposed and existing telescopes.
% - For antenna arrays, provide specific infrastructure required such as
%   concrete pad size, communications buildings, etc. for each element in
%   the array. For telescope mirrors, describe infrastructure needed for
%   mirror maintenance, e.g. coating facilities.
% - Describe atmospheric and radio frequency interference (RFI)
%   characteristics of the site insofar as they would affect observations
%   with the concept being proposed.

\section{Operations \& Observation Strategy}
\label{sec:ops}

% - For instrument operations, provide a brief functional description of
%   operational modes, and calibration schemes. This can be documented in
%   the Operations Section. Describe the level of complexity associated
%   with analyzing the data to achieve the scientific objectives of the
%   investigation. Describe the types of data (e.g. bits, images).

FOBOS is a general-purpose, facility-class instrument that is
expected to integrate into the existing operational structure of the
W.~M.~Keck Observatory. Other instruments will share use of the Keck
II Nasmyth port used by FOBOS (e.g., KCWI), such that the focal-plane
module will be regularly rolled in and out of position, using the
rails pictured in Fig. \ref{fig:layout}. Given their size and our
emphasis on spectrograph stability, however, FOBOS's spectrographs
will be housed in a permanent, temperature-controlled stucture on the
Nasymth deck.

FOBOS emphasizes flexibility in focal plane sampling, providing
multiple focal-plane modules with different sampling options. These
modules will be stored with the spectrograph structure and exchanged
during a day-time procedure; i.e., a single focal-plane sampling
option will be used in a given night. For a given focal-plane
sampling mode, FOBOS will support various observing modes ranging
from direct, PI-led control of the system to an automated, AI-led
merging of multiple observing programs (see MAISTRO below).

% - Provide a description of the science operations. For example:
%   pre-observational planning, data processing/reduction required,
%   coordination with other facilities.
% - Summarize key software development and any science development
%   required.
% - Summarize any archiving requirements. 

\subsection{Data Management System}

FOBOS is a complex instrument, both in terms of its active hardware
systems and in how it will be used. As such, FOBOS will require a
sophisticated data management system, which governs all software
related to planning and executing FOBOS observations, as well as the
collection and processing of the resulting scientific data.
Commensurate with FOBOS's development phase, we outline this system
below. We note here that this system specifically does not include
any software related to the operation or control of the FOBOS
mechanisms or their components. Operational software is handled by a
separate WBS element. The FOBOS data-management system is broken into
five main interdependent subsystems and an overarching application
program interface (API), {\it the Manager}, that allows for
communication between these systems. These five sub-systems are:

~\comment{emphasize partners more here}

\begin{enumerate}
%
\item {\it The Doctor}: FOBOS will maintain a database of metadata
that monitors the current state of the instrument, a historical
record of the state of the site, telescope, and instrument at the
time of the observations, and figures-of-merit for the quality of the
data produced.
%
\item {\it The Producer}: FOBOS will provide a holistic software
package to plan full, multiple-pointing observing programs. This is
divided into two subsystems: (1) {\it MAISTRO}, Modular Artificial
Intelligence System for Target Reallocation and Observing, an
``artificial intelligence'' (AI) targeting system that will learn
optimization strategies for user-controlled and multi-program target
assignment, and (2) {\it the Composer}, an instrument simulation
suite that produces realistic raw data products that can be processed
through the data-reduction and data-analysis software.
%
\item {\it The Accountant}: FOBOS will provide a data-reduction
software package that provides both quick-look and science-ready,
calibrated spectra.  This system will inherit from our teams historical connection to fiber-based packages built for SDSS and DESI.
%
\item {\it The Alchemist}: FOBOS will provide an automated
data-analysis software package that provides a set of relatively
simple, high-level measurements made from the spectra (e.g.,
redshift), building on our experience with the MaNGA Data-Analysis
Pipeline.
%
\item {\it The Curator}: FOBOS will maintain an archive of all data,
from observation-specific metadata and raw detector output to reduced
and higher-level data products. This is divided into two subsystems,
the roughly static data archive provided by the Keck Observatory
Archive (KOA) and a dynamic interface and science platform for
visualization and analysis. The latter will be done in collaboration
with NSF's OIR lab and its Community Science and Data Center (CSDC).
%
\end{enumerate}


% From Keck?
% - Please provide operations plans or documents (Concept of Operations). 
% - Provide a description of the facility operations. For example: number
%   of staff required, position types, and 24-hour operation requirements.
% - Discuss observatory efficiency, e.g. the impact of maintenance and
%   engineering and calibration time on faction of science time
%   availability.
% - Describe scope of engineering activities and time (day or night)
%   needed to maintain calibration and health of telescope/array and
%   instruments.
% - Describe any high-level safety policies that will be required for
%   items of particular safety concern.

\bigskip
\section{Programmatic Issues \& Schedule}

\begin{figure}[h!]
\vskip -0.1in
\includegraphics[width=\textwidth]{org_chart_v2.pdf}
\caption{\small FOBOS management and leadership structure.}
\label{fig:org}
\end{figure}


% - Please provide a programmatic overview that describes the structure of
%   the overall organization including any international partners or
%   university partners etc. and any money or hardware they are providing.
%   Clearly indicate schedule and costs to date highlighting what has
%   already been delivered and/or clearly capturing progress to date, if
%   applicable.

FOBOS is an international collaboration with partner institutions in Arizona, California, Hawaii, Kentucky, Pennsylvania, Washington, and Australia.  The central project office is centered at U.C.\ Santa Cruz under the University of California Observatories.  FOBOS continues to actively recruit partners and expects the collaboration will grow over time.  The organizational chart and management structure is shown in Figure \ref{fig:org}.  Each partner institution is described further below:

\medskip
\noindent{\textbf{University of California Observatories.}}  The University of California Observatories (UCO) manages a
world-renowned facility on the University of California, Santa Cruz, campus for the design, construction, and testing
of astronomical instrumentation.  With a staff of leading optical designers, engineers, and instrument scientists, UCO
has a long heritage of producing state-of-the-art instrumentation, including many spectrometers (e.g., DEIMOS), as well
as controls software for the Lick and Keck Observatories.  The recent delivery of K1DM3\footnote{K1DM3: Keck 1
Deployable Tertiary Mirror.} illustrates the close relationship between WMKO and UCO, which allows us to leverage
detailed knowledge of the observatory structure, protocols, interfaces, software and systems requirements, and
instrument deliverables.

\noindent{\textbf{W.~M.~Keck Observatory.}} WMKO has provided funding as well as technical guidance for initial stages of FOBOS development and its interface to
the observatory.  FOBOS ranks as one of WMKO's top priorities in the coming decade, as encapsulated in its 2016
Strategic Plan and evidenced by its commitment to make 30 nights available to the U.S.\ community over this MSIP
proposal period as a way to deepen community involvement in Keck during the FOBOS design phase. WMKO personnel play a
key role in the proposed work through joint design of physical interfaces, safety protocols, the calibration systems,
operational modes, and software systems both for operations and data management.  Initial work has already identified
FOBOS placement and mounting options that allow it to integrate into Keck's planning for its evolving instrument suite.

\noindent{\textbf{SSL, LBNL, and DESI Fiber Lab.}} C.~Poppett serves as one of two FOBOS instrument scientists,
bringing significant expertise and facility resources from the DESI fiber system build.  Facilities developed for the
DESI fiber system include a precision optical bonding bench that facilitated the manufacture and test of 6000 fiber
optic assemblies for DESI. DESI also connected the focal plane fibers to the spectrograph slits via a custom splicing
station.  FOBOS, through a partnership with Berkeley's Space Science's Lab (SSL), will be able to utilize much of the
same lab equipment, developed expertise, detailed training documentation and procedures.

\noindent{\textbf{NSF OIR Lab.}}  NSF’s OIR Lab is an integrated national center recently created through the
combination of NOAO, Gemini, and LSST Operations. The Lab brings significant experience connecting with and serving the
U.S. astronomical community. FOBOS will utilize this background in developing open-access models and community key
programs. NSF’s OIR Lab is also investing significantly in building data servicing platforms for public-facing data
sets from LSST, DESI, and elsewhere. The Lab is a key partner in developing tools for science-ready interaction with
high-level FOBOS data products.

\noindent{\textbf{Australian Astronomical Observatory.}} The Australian Astronomical Optics (AAO, previously Australian
Astronomical Observatory) has worked with the FOBOS team during conceptual design to develop designs for the CLADC and
confirm the feasibility of Starbugs at the FOBOS focal plane.  The Starbugs fiber positioners have been under
development at AAO for nearly 15 years and have recently been deployed on-sky on a new instrument called TAIPAN
\cite{staszak16}.  Beginning science operations this year, TAIPAN demonstrates that the level of maturity attained by
the Starbugs technology makes it highly attractive to an instrument like FOBOS beginning its preliminary design phase.



\noindent{\textbf{Swinburne University.}} Swinburne is a long-time Keck Observatory partner and will be prototyping micro-optics assemblies for FOBOS fibers.  Swinburne is pursuing Australian ``LIEF'' funding opportunities to increase in-kind FOBOS contributions in 2020.


\noindent{\textbf{CMU \& LSST Data Science.}} The Carnegie Mellon University Machine Learning and Statistics
departments are leading institutes in the field of data science.  Project members Mandelbaum and Rau are long time
collaborators of the CMU Machine Learning department.  CMU Department of Statistics professor C.~Schafer is Co-Chair of
LSST's Informatics and Statistics Science Collaboration (ISSC).


\noindent{\textbf{UW \& LSST Operations.}} The University of Washington (UW) is an important LSST institution with significant planning, operations and data-support personnel that we will leverage for the design of similar FOBOS systems.  This not only provides valuable expertise but deepens the link between FOBOS and one of its most critical targeting data sets in LSST.


\noindent{\textbf{UC Campuses.}} D.~Weisz (UC Berkeley) is driving the Local Group and M31 satellites science definition and requirements.  Team member M.~Cooper (UC Irvine) lends significant expertise and resources in contributing to FOBOS survey planning and data reduction efforts.  At UCLA, M.~Rich co-leads the FOBOS Andromeda science program while A.~Shapley is involved in science program definition for $z \sim 2$ galaxy evolution studies.  At UC Riverside, B.~Siana is helping define the monolithic IFU design drivers and G.~Becker to motivating high-multiplex observations at high redshift.  J.~Hennawi (UCSB) is contributing to the IGM tomography and quasar light-echo science.  At UC Santa Cruz, J.~Burchett leads the FOBOS galaxy program while R.~Guhathurkata and C.~Rockosi are helping define the M31 survey program.  J.~Brodie, also at UCSC, is contributing to requirements definition for IFU science of nearby diffuse galaxies.

\noindent{\textbf{IPAC.}} The Infrared Processing \& Analysis Center is a major partner in the Keck Observatory Archive (KOA) which will serve raw and reduced FOBOS data.  IPAC is exploring cross-referencing tools to connect FOBOS planning and observations to available photometry catalogs.

\noindent{\textbf{Caltech.}} Team member, E.~Kirby, is a major contributor to the FOBOS M31 program definition and bridges FOBOS development with Subaru's PFS, of which Caltech is a partner.


% - Please describe any funding availability challenges and the total
%   impact it had, or may have, to cost and schedule.  Has all required
%   funding been committed?  If not, please explain.  Highlight in-kind
%   contributions in the past and planned from partners.  Clearly describe
%   what the NSF/DOE/Fed Gov funding request is and what it buys.
% - Please describe any unexpected challenges, key risks, and/or current
%   outstanding risks that require(d) significant mitigation which
%   affect/affected cost & schedule.
% - Describe the current top 3 risks to the development of the facility,
%   and proposed mitigation strategies. Please provide a top-level
%   schedule.

\section{Cost}
\label{sec:cost}

% - Provide a high-level facility unique Work Breakdown Structure (WBS)
%   with definitions. 
% - Fill out the cost estimate tables below using the broad bins provided.
%   Note the tables are divided into US Only and with International
%   Partners.  
% - Provide a basis of estimate for project WBS elements at the highest
%   levels (i.e., systems and major subsystems, as indicated in the cost
%   tables below), including internal overhead rates that are applied to
%   costs for labor.  Describe approach or methodology for highest cost
%   element estimates such as analogies, models, expert judgement,
%   construction bids, etc.  Describe cost assumptions made for each
%   element such as any production learning curves, labor and major
%   procurements.  Identify major subcontracted procurements and
%   associated vendors if available.  The “Prior” column is meant to be
%   actual costs incurred. 


\newpage

\begin{multicols}{2}
\scriptsize
\bibliographystyle{apj}
\bibliography{references}
\end{multicols}

\end{document}


