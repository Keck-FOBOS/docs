%\documentclass[11pt,letterpaper]{article}
\documentclass[oneside,11pt]{amsart}

%\usepackage{a4wide}
%\usepackage{epsfig}
%\usepackage{psfig}
\usepackage{graphicx}
\usepackage{natbib,latexsym,url,enumitem,pdfpages}
\usepackage{color}
\usepackage{wrapfig}
\usepackage{caption}

\captionsetup{
    justification=justified,
    margin=0pt,
    font=small}

% Some fancy commenting
\definecolor{todo}{RGB}{200,0,0}
\newcommand{\comment}[2][todo]{{\color{#1}[[{\bf #2}]]}}

% Challenge counter
\newcounter{chalno}
\newcommand{\chal}[1]{\refstepcounter{chalno}\label{#1}}

% User commands
\makeatletter
\let\jnl@style=\rm
\def\ref@jnl#1{{\jnl@style#1}}

\def\ref@jnl#1{{\jnl@style#1}}% 
\newcommand\aj{\ref@jnl{AJ}}%        % Astronomical Journal 
\newcommand\araa{\ref@jnl{ARA\&A}}%  % Annual Review of Astron and Astrophys 
\newcommand\apj{\ref@jnl{ApJ}}%    % Astrophysical Journal ++
\newcommand\apjl{\ref@jnl{ApJL}}     % Astrophysical Journal, Letters 
\newcommand\apjs{\ref@jnl{ApJS}}%    % Astrophysical Journal, Supplement 
\newcommand\ao{\ref@jnl{ApOpt}}%   % Applied Optics ++
\newcommand\apss{\ref@jnl{Ap\&SS}}%  % Astrophysics and Space Science 
\newcommand\aap{\ref@jnl{A\&A}}%     % Astronomy and Astrophysics 
\newcommand\aapr{\ref@jnl{A\&A~Rv}}%  % Astronomy and Astrophysics Reviews 
\newcommand\aaps{\ref@jnl{A\&AS}}%    % Astronomy and Astrophysics, Supplement 
\newcommand\azh{\ref@jnl{AZh}}%       % Astronomicheskii Zhurnal 
\newcommand\baas{\ref@jnl{BAAS}}%     % Bulletin of the AAS 
\newcommand\icarus{\ref@jnl{Icarus}}% % Icarus
\newcommand\jrasc{\ref@jnl{JRASC}}%   % Journal of the RAS of Canada 
\newcommand\memras{\ref@jnl{MmRAS}}%  % Memoirs of the RAS 
\newcommand\mnras{\ref@jnl{MNRAS}}%   % Monthly Notices of the RAS 
\newcommand\pra{\ref@jnl{PhRvA}}% % Physical Review A: General Physics ++
\newcommand\prb{\ref@jnl{PhRvB}}% % Physical Review B: Solid State ++
\newcommand\prc{\ref@jnl{PhRvC}}% % Physical Review C ++
\newcommand\prd{\ref@jnl{PhRvD}}% % Physical Review D ++
\newcommand\pre{\ref@jnl{PhRvE}}% % Physical Review E ++
\newcommand\prl{\ref@jnl{PhRvL}}% % Physical Review Letters 
\newcommand\pasp{\ref@jnl{PASP}}%     % Publications of the ASP 
\newcommand\pasj{\ref@jnl{PASJ}}%     % Publications of the ASJ 
\newcommand\qjras{\ref@jnl{QJRAS}}%   % Quarterly Journal of the RAS 
\newcommand\skytel{\ref@jnl{S\&T}}%   % Sky and Telescope 
\newcommand\solphys{\ref@jnl{SoPh}}% % Solar Physics 
\newcommand\sovast{\ref@jnl{Soviet~Ast.}}% % Soviet Astronomy 
\newcommand\ssr{\ref@jnl{SSRv}}% % Space Science Reviews 
\newcommand\zap{\ref@jnl{ZA}}%       % Zeitschrift fuer Astrophysik 
\newcommand\nat{\ref@jnl{Nature}}%  % Nature 
\newcommand\iaucirc{\ref@jnl{IAUC}}% % IAU Cirulars 
\newcommand\aplett{\ref@jnl{Astrophys.~Lett.}}%  % Astrophysics Letters 
\newcommand\apspr{\ref@jnl{Astrophys.~Space~Phys.~Res.}}% % Astrophysics Space Physics Research 
\newcommand\bain{\ref@jnl{BAN}}% % Bulletin Astronomical Institute of the Netherlands 
\newcommand\fcp{\ref@jnl{FCPh}}%   % Fundamental Cosmic Physics 
\newcommand\gca{\ref@jnl{GeoCoA}}% % Geochimica Cosmochimica Acta 
\newcommand\grl{\ref@jnl{Geophys.~Res.~Lett.}}%  % Geophysics Research Letters 
\newcommand\jcp{\ref@jnl{JChPh}}%     % Journal of Chemical Physics 
\newcommand\jgr{\ref@jnl{J.~Geophys.~Res.}}%     % Journal of Geophysics Research 
\newcommand\jqsrt{\ref@jnl{JQSRT}}%   % Journal of Quantitiative Spectroscopy and Radiative Trasfer 
\newcommand\memsai{\ref@jnl{MmSAI}}% % Mem. Societa Astronomica Italiana 
\newcommand\nphysa{\ref@jnl{NuPhA}}%     % Nuclear Physics A 
\newcommand\physrep{\ref@jnl{PhR}}%       % Physics Reports 
\newcommand\physscr{\ref@jnl{PhyS}}%        % Physica Scripta 
\newcommand\planss{\ref@jnl{Planet.~Space~Sci.}}%  % Planetary Space Science 
\newcommand\procspie{\ref@jnl{Proc.~SPIE}}%      % Proceedings of the SPIE 

\newcommand\actaa{\ref@jnl{AcA}}%  % Acta Astronomica
\newcommand\caa{\ref@jnl{ChA\&A}}%  % Chinese Astronomy and Astrophysics
\newcommand\cjaa{\ref@jnl{ChJA\&A}}%  % Chinese Journal of Astronomy and Astrophysics
\newcommand\jcap{\ref@jnl{JCAP}}%  % Journal of Cosmology and Astroparticle Physics
\newcommand\na{\ref@jnl{NewA}}%  % New Astronomy
\newcommand\nar{\ref@jnl{NewAR}}%  % New Astronomy Review
\newcommand\pasa{\ref@jnl{PASA}}%  % Publications of the Astron. Soc. of Australia
\newcommand\rmxaa{\ref@jnl{RMxAA}}%  % Revista Mexicana de Astronomia y Astrofisica

%% added feb 9, 2016
\newcommand\maps{\ref@jnl{M\&PS}}% Meteoritics and Planetary Science
\newcommand\aas{\ref@jnl{AAS Meeting Abstracts}}% American Astronomical Society Meeting Abstracts
\newcommand\dps{\ref@jnl{AAS/DPS Meeting Abstracts}}% American Astronomical Society/Division for Planetary Sciences Meeting Abstracts



\let\astap=\aap 
\let\apjlett=\apjl 
\let\apjsupp=\apjs 
\let\applopt=\ao 



\DeclareRobustCommand{\gtrsim}{%
\mathrel{\hskip-.5em\begin{array}{c}>\\[-8pt]\sim\end{array}\hskip-.5em}}
\DeclareRobustCommand{\lesssim}{%
\mathrel{\hskip-.5em\begin{array}{c}<\\[-8pt]\sim\end{array}\hskip-.5em}}

\pretolerance=10000
\textwidth=6.4in
\textheight=8.95in
\voffset = 0.in
%\voffset = -0.3in  % For my printer
\topmargin=0.0in
\headheight=0.00in
\hoffset = 0.0in
%\hoffset = 0.33in  %  For my printer
\headsep=0.00in
\oddsidemargin=0in
\evensidemargin=0in
\parindent=2em
\parskip=0.2ex
 
\renewcommand{\baselinestretch}{1.03}

\special{papersize=8.5in,11in}

\newcommand{\markus}{\textcolor{green}}

\setlength{\parskip}{0.6 ex plus 0.4ex minus 0.2ex} \flushbottom
\pagestyle{plain} 

\begin{document}
% \thispagestyle{empty}

\pagenumbering{arabic}

\vspace*{-1.5cm}

\section*{Partner Organizations}

\noindent Our proposal includes eight subawards:
\begin{enumerate}
%
\item {\bf Institute for Scientist \& Engineer Educators, University
of California, Santa Cruz (PI Hunter):} Hunter and ISEE staff will
organize the participation of eight undergraduate interns in the
pre-existing Akamai program to work on FOBOS-related projects, four
per summer during the grant duration. They will also organize the
participation of four graduate students in the Professional
Development program, two per academic year.
%
\item {\bf Space Sciences Laboratory, University of California,
Berkeley (PI Poppett):} Berkeley/SSL will provide the design of the
optical components that feed the FOBOS spectrographs, including the
compensating lateral atmospheric dispersion corrector (CLADC), fold
plate, and the fiber$+$microlens system.
%
\item {\bf Australian Astronomical Optics - Macquarie University (PI
Lawrence):} AAO-Macquarie will provide the design for the robotic
positioning system based on the Starbugs technology.
%
\item {\bf W.~M.~Keck Observatory (PI Kassis):} W.~M.~Keck
Observatory (WMKO) is the site that will ultimately host FOBOS. WMKO
will provide the mechanical design and integration plan for the
retractable FOBOS focal-plane system and the permanent spectrograph
structure. WMKO will also provide the mechanical design of the FOBOS
calibration system.
%
\item {\bf University of Kentucky, Department of Physics \& Astronomy
(PI Yan):} Professor Renbin Yan will provide the optical design of
the calibration system and the nominal calibration strategy for
standard instrument operation modes.
%
\item {\bf NSF's Optical-Infrared Laboratory (PI Bolton):} NSF's OIR
Lab will develop observing planning tools that facilitate the design
of observational programs to be executed via FOBOS's open-access
model. They will also design the data systems that enable the broader
US community to capitalize on FOBOS public data via online science
platforms.
%
\item {\bf Carnegie Mellon University, Department of Physics (PI
Mandelbaum):} Professors Rachel Mandelbaum and Chad Schafer will lead
the design of software tools that take advantage of state-of-the-art
statistical techniques (Bayesian optimization and machine learning)
to optimize the definition and execution of FOBOS's key observing
programs.
%
\item {\bf University of Washington, Department of Astronomy (PI
Williams):} Capitalizing on local expertise for NSF's Large Synoptic
Survey Telescope (LSST), Ben Williams will lead the design of the
data system that stores, monitors, and predicts the health of the
FOBOS instrument. He will also help develop significant components of
the instrument simulator and data-reduction systems.
%
\end{enumerate}

\end{document}


