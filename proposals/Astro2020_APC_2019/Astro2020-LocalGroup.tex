%%%%
% -- Local Group Science Cases
% --     FOBOS Keck White Paper 2019
%%%%

\subsection{Assembly History of the Local Group}
%Unraveling the Formation History of our Local Group of Galaxies}
\label{sec:localgroup}

Studies of individual stars in the Milky Way (MW), Magellanic Clouds, Andromeda (M31), Triangulum galaxy (M33), and
numerous dwarf satellites provide an exquisitely detailed look at specific examples of galaxy assembly and evolution.
While Gaia provides on-sky motions and photometry for 1.7 billion stars in the MW, fewer than 10\%, 0.3\%, and 0.1\% of
stars will have a full complement of astrometrics and kinematics, basic stellar parameters, and chemical abundances,
respectively.  Moreover, Gaia distance errors increase quadratically with distance.  Spectroscopy with APOGEE, the
Milky Way Mapper, and WEAVE provide supporting wide-field data sets but accounting for fainter stars requires
FOBOS-like sensitivity \citep[see][]{dey19,sanderson19}.  By carefully exploiting the overlap in these data sets, FOBOS
can link high-resolution and robust stellar information from brighter targets to stars that can only be characterized
by photometry.  This would enable data-driven models capable of providing photometric estimates of stellar
parameters (temperature, surface gravity, metallicity, and alpha-element abundance) for {\it all} stars in the Gaia
dataset  \citep[see][]{2015ApJ...808...16N, 2018arXiv180401530T, 2018arXiv180803278T}.

Of particular interest is the ability of future imaging surveys to increase the census of stellar streams and other
substructure by a hundredfold.  The stars in these structures are faint, however, and easily confused with background
galaxies in ground-based photometry.  With spectrocopic reference samples from FOBOS, the goal is to photometrically
reconstruct the star-formation histories of disrupted satellites and compare them with dynamical models to constrain
assembly histories and enclosed mass constraints \citep[e.g.,][]{2017ApJ...836..234S}.


Performing a similar analysis on the M31 halo is highly desireable but more challenging because individual
main-sequence stars at the distance of M31 are too faint for 10m telescopes.  Thus spectroscopic training efforts must
focus on giant stars in the M31 halo and be calibrated with hydrodynamical simulations that account for M31's differing
formation history  \citep[e.g.][]{2005MNRAS.356.1071R,li19}.



%  Complementing faint {\it Gaia} targets, FOBOS will
% enable machine-learning for photometric stellar parameter recovery
% in service of direct probes
% of radial migration, disk heating, and other assembly processes in
% the MW and M31.


% FOBOS spectroscopy of
% these structures will, e.g., constrain stream orbits and the total
% mass they enclose . These data will also
% provide age and chemical composition measurements either directly or
% via targeted machine-learning applications . FOBOS is well-suited to these observations by
% dramatically improving the survey speed over DEIMOS (by an order of
% mangitude or more) and will remain sensitive enough to push toward
% the main-sequence turn-off of MW substructures. Compared to the MW,
% studies of M31's disk and halo are more suited for dedicated FOBOS
% programs. Indeed, the dramatic improvement in survey speed will,
% e.g., yield a direct measurements of secular processes active in
% M31's disk, such as radial migration and vertical disk heating
% \citep{2013ApJ...779..103D, 2015ApJ...803...24D,
% 2019ApJ...871...11Q}. Even so, {\it Gaia} targets, particularly at
% larger distances and higher target densities in the bulge, bar and
% inner disk, will prove uniquely suitable for FOBOS spectroscopy.

%\comment{Guhathakurta, Rockosi, Weisz}

% \chal{mwhalo} 
% %
% \item[] {\textsf {\large  Data-Science Challenge \ref{mwhalo}: The
% chemical evolution and assembly history of the MW stellar halo.}}  Using current MW halo models, we will simulate FOBOS
% stellar spectroscopy of main-sequence turn-off
% and red-giant stars in these substructures within the MW that also
% leverages existing data from, e.g., APOGEE and H3.  We will build
% data-driven models based on these data to measure stellar parameters
% (temperature, surface gravity, metallicity, and alpha-element abundance)
% for all halo stars with LSST+2MASS+WISE+WFIRST multi-band photometry,
% allowing us to reconstruct the star-formation history of each disrupted
% satellite. These will be combined with dynamical data and compared with
% cosmological simulations to build a generative model for the assembly
% history of the MW stellar halo.

% \chal{m31} 
% %
% \item[] {\textsf {\large Data-Science Challenge \ref{m31}: The
% differential chemical evolution of M31 and MW.}}  A natural extension of
% Data-Science Challenge \ref{mwhalo} is to perform the same analysis for the
% halo of M31.  However, we cannot expect to obtain high-quality spectra
% of individual main-sequence stars at the distance of M31 with FOBOS.
% Moreover, training a chemical evolution model using spectra of Milky Way
% stars may lead to systematic errors:  The Milky Way and Andromeda have
% distinct evolutionary histories \citep[e.g.][]{2005MNRAS.356.1071R},
% despite being relatively similar in many other respects.  We will
% therefore obtain deep observations of giant stars in the M31 halo to
% drive a machine-learning algorithm that combines a model of the MW halo
% with results from cosmological hydrodynamical simulations to constrain
% the differential history of the MW and M31 stellar halos.

% \chal{gaia} 
% %
% \item[] {\textsf {\large Data-Science Challenge \ref{gaia}: Stellar
% parameter determinations for a billion stellar spectra.}} While
% providing on-sky motions and photometry for 1.7 billion stars in the MW,
% fewer than 10\%, 0.3\%, and 0.1\% of stars will have a full complement
% of astrometrics and kinematics, basic stellar parameters, and chemical
% abundances, respectively.  Moreover, Gaia distance errors increase
% quadratically with distance.  To realize Gaia's full potential, we will
% design FOBOS training sets that, when combined with high-resolution
% datasets from, e.g., APOGEE, WEAVE, will allow us to build data-driven
% models of the absolute magnitude (yielding distance modulus),
% temperature, surface-gravity, and stellar abundance for {\it all} stars
% in the Gaia dataset.  These data will allow us to isolate coeval
% populations in the Galactic disk that can be combined with very
% high-resolution simulations of the Milky Way to provide a detailed
% evolutionary history of our Galactic home.

