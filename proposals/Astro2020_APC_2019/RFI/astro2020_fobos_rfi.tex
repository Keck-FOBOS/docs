%\documentclass[11pt,letterpaper]{article}
\documentclass[oneside,11pt]{amsart}

%\usepackage{a4wide}
%\usepackage{epsfig}
%\usepackage{psfig}
\usepackage{graphicx}
\usepackage{natbib,latexsym,url,enumitem,pdfpages}
\usepackage{color}
\usepackage{wrapfig}
\usepackage[belowskip=-10pt,aboveskip=0pt]{caption}

\captionsetup{
    justification=justified,
    margin=0pt,
    font=small}

%%%%%%%%%%%%%%%%%%%%%%%%%%%%%%%%%%%%%%%%%%%%%%%%%%%%%%%%%%%%%%%%%%%%%%%%
% Allow for the okina; thanks to:
% https://tex.stackexchange.com/questions/424535/how-to-type-a-proper-hawai%CA%BBian-%CA%BBokina

\usepackage[utf8]{inputenc}
\usepackage{newunicodechar}
%\usepackage{libertine}

\DeclareRobustCommand{\okina}{%
  \raisebox{\dimexpr\fontcharht\font`A-\height}{%
    \scalebox{0.8}{`}%
  }%
}
\newunicodechar{ʻ}{\okina}
\newcommand{\hawaii}{Hawaiʻi}
%%%%%%%%%%%%%%%%%%%%%%%%%%%%%%%%%%%%%%%%%%%%%%%%%%%%%%%%%%%%%%%%%%%%%%%%

\newcommand{\arcsec}{\mbox{$^{\prime\prime}$}}
\newcommand{\gt}{$>$}

% Some fancy commenting
\definecolor{todo}{RGB}{200,0,0}
\newcommand{\comment}[2][todo]{{\color{#1}[[{\bf #2}]]}}

% Challenge counter
\newcounter{chalno}
\newcommand{\chal}[1]{\refstepcounter{chalno}\label{#1}}

% User commands
\makeatletter
\let\jnl@style=\rm
\def\ref@jnl#1{{\jnl@style#1}}

\def\ref@jnl#1{{\jnl@style#1}}% 
\newcommand\aj{\ref@jnl{AJ}}%        % Astronomical Journal 
\newcommand\araa{\ref@jnl{ARA\&A}}%  % Annual Review of Astron and Astrophys 
\newcommand\apj{\ref@jnl{ApJ}}%    % Astrophysical Journal ++
\newcommand\apjl{\ref@jnl{ApJL}}     % Astrophysical Journal, Letters 
\newcommand\apjs{\ref@jnl{ApJS}}%    % Astrophysical Journal, Supplement 
\newcommand\ao{\ref@jnl{ApOpt}}%   % Applied Optics ++
\newcommand\apss{\ref@jnl{Ap\&SS}}%  % Astrophysics and Space Science 
\newcommand\aap{\ref@jnl{A\&A}}%     % Astronomy and Astrophysics 
\newcommand\aapr{\ref@jnl{A\&A~Rv}}%  % Astronomy and Astrophysics Reviews 
\newcommand\aaps{\ref@jnl{A\&AS}}%    % Astronomy and Astrophysics, Supplement 
\newcommand\azh{\ref@jnl{AZh}}%       % Astronomicheskii Zhurnal 
\newcommand\baas{\ref@jnl{BAAS}}%     % Bulletin of the AAS 
\newcommand\icarus{\ref@jnl{Icarus}}% % Icarus
\newcommand\jrasc{\ref@jnl{JRASC}}%   % Journal of the RAS of Canada 
\newcommand\memras{\ref@jnl{MmRAS}}%  % Memoirs of the RAS 
\newcommand\mnras{\ref@jnl{MNRAS}}%   % Monthly Notices of the RAS 
\newcommand\pra{\ref@jnl{PhRvA}}% % Physical Review A: General Physics ++
\newcommand\prb{\ref@jnl{PhRvB}}% % Physical Review B: Solid State ++
\newcommand\prc{\ref@jnl{PhRvC}}% % Physical Review C ++
\newcommand\prd{\ref@jnl{PhRvD}}% % Physical Review D ++
\newcommand\pre{\ref@jnl{PhRvE}}% % Physical Review E ++
\newcommand\prl{\ref@jnl{PhRvL}}% % Physical Review Letters 
\newcommand\pasp{\ref@jnl{PASP}}%     % Publications of the ASP 
\newcommand\pasj{\ref@jnl{PASJ}}%     % Publications of the ASJ 
\newcommand\qjras{\ref@jnl{QJRAS}}%   % Quarterly Journal of the RAS 
\newcommand\skytel{\ref@jnl{S\&T}}%   % Sky and Telescope 
\newcommand\solphys{\ref@jnl{SoPh}}% % Solar Physics 
\newcommand\sovast{\ref@jnl{Soviet~Ast.}}% % Soviet Astronomy 
\newcommand\ssr{\ref@jnl{SSRv}}% % Space Science Reviews 
\newcommand\zap{\ref@jnl{ZA}}%       % Zeitschrift fuer Astrophysik 
\newcommand\nat{\ref@jnl{Nature}}%  % Nature 
\newcommand\iaucirc{\ref@jnl{IAUC}}% % IAU Cirulars 
\newcommand\aplett{\ref@jnl{Astrophys.~Lett.}}%  % Astrophysics Letters 
\newcommand\apspr{\ref@jnl{Astrophys.~Space~Phys.~Res.}}% % Astrophysics Space Physics Research 
\newcommand\bain{\ref@jnl{BAN}}% % Bulletin Astronomical Institute of the Netherlands 
\newcommand\fcp{\ref@jnl{FCPh}}%   % Fundamental Cosmic Physics 
\newcommand\gca{\ref@jnl{GeoCoA}}% % Geochimica Cosmochimica Acta 
\newcommand\grl{\ref@jnl{Geophys.~Res.~Lett.}}%  % Geophysics Research Letters 
\newcommand\jcp{\ref@jnl{JChPh}}%     % Journal of Chemical Physics 
\newcommand\jgr{\ref@jnl{J.~Geophys.~Res.}}%     % Journal of Geophysics Research 
\newcommand\jqsrt{\ref@jnl{JQSRT}}%   % Journal of Quantitiative Spectroscopy and Radiative Trasfer 
\newcommand\memsai{\ref@jnl{MmSAI}}% % Mem. Societa Astronomica Italiana 
\newcommand\nphysa{\ref@jnl{NuPhA}}%     % Nuclear Physics A 
\newcommand\physrep{\ref@jnl{PhR}}%       % Physics Reports 
\newcommand\physscr{\ref@jnl{PhyS}}%        % Physica Scripta 
\newcommand\planss{\ref@jnl{Planet.~Space~Sci.}}%  % Planetary Space Science 
\newcommand\procspie{\ref@jnl{Proc.~SPIE}}%      % Proceedings of the SPIE 

\newcommand\actaa{\ref@jnl{AcA}}%  % Acta Astronomica
\newcommand\caa{\ref@jnl{ChA\&A}}%  % Chinese Astronomy and Astrophysics
\newcommand\cjaa{\ref@jnl{ChJA\&A}}%  % Chinese Journal of Astronomy and Astrophysics
\newcommand\jcap{\ref@jnl{JCAP}}%  % Journal of Cosmology and Astroparticle Physics
\newcommand\na{\ref@jnl{NewA}}%  % New Astronomy
\newcommand\nar{\ref@jnl{NewAR}}%  % New Astronomy Review
\newcommand\pasa{\ref@jnl{PASA}}%  % Publications of the Astron. Soc. of Australia
\newcommand\rmxaa{\ref@jnl{RMxAA}}%  % Revista Mexicana de Astronomia y Astrofisica

%% added feb 9, 2016
\newcommand\maps{\ref@jnl{M\&PS}}% Meteoritics and Planetary Science
\newcommand\aas{\ref@jnl{AAS Meeting Abstracts}}% American Astronomical Society Meeting Abstracts
\newcommand\dps{\ref@jnl{AAS/DPS Meeting Abstracts}}% American Astronomical Society/Division for Planetary Sciences Meeting Abstracts



\let\astap=\aap 
\let\apjlett=\apjl 
\let\apjsupp=\apjs 
\let\applopt=\ao 



\DeclareRobustCommand{\gtrsim}{%
\mathrel{\hskip-.5em\begin{array}{c}>\\[-8pt]\sim\end{array}\hskip-.5em}}
\DeclareRobustCommand{\lesssim}{%
\mathrel{\hskip-.5em\begin{array}{c}<\\[-8pt]\sim\end{array}\hskip-.5em}}


\pretolerance=10000
\textwidth=6.4in
\textheight=8.95in
\voffset = 0.in
%\voffset = -0.3in  % For my printer
\topmargin=0.0in
\headheight=0.00in
\hoffset = 0.0in
%\hoffset = 0.33in  %  For my printer
\headsep=0.00in
\oddsidemargin=0in
\evensidemargin=0in
\parindent=2em
\parskip=0.2ex
 
\renewcommand{\baselinestretch}{1.03}

\special{papersize=8.5in,11in}

\newcommand{\markus}{\textcolor{green}}

\setlength{\parskip}{0.6 ex plus 0.4ex minus 0.2ex} \flushbottom
\pagestyle{plain} 

\begin{document}
% \thispagestyle{empty}

\pagenumbering{arabic}

\vspace*{-1.5cm}

\centerline{\textsf {\Large FOBOS: A Next-Generation Spectroscopic Facility}}
\centerline{\textsf {\large Response to Astro2020 Request for Information}}

\setcounter{page}{1}

\section*{Executive Summary}

% - Summarize your science objectives and your technical implementation at
%   a high level.

\noindent{\bf Science objectives:} FOBOS, the Fiber-Optic Broadband
Optical Spectrograph, is a facility-class, general-purpose
spectrograph for the 10m Keck II telescope. It emphasizes UV
sensitivity (up to the atmospheric limit), high multiplex, multiple
focal-plane sampling formats, and near Poissonian-level performance
over extremely long ($\sim$100 hr) integration times. These emphases
establish FOBOS's uniqueness among the suite of spectrographs coming
online for 8-10m class telescopes in the next decade. FOBOS is
specifically built to enable the deep spectroscopic follow-up of
upcoming large-scale imaging surveys (LSST, WFIRST, and Euclid)
identified as a significant need of the astronomy community in the US
and beyond. For example, capitalizing on the combined $\approx$\$4B
the US is investing in these imaging surveys, FOBOS can increase
LSST's dark-energy figure-of-merit by 40\% (Newman et al.\ 2015).

\smallskip

\noindent{\bf Technical implementation:} FOBOS is a fiber-fed,
fixed-format spectrograph. Its primary systems include (1) a
compensating lateral atmospheric dispersion corrector (CLADC), (2) a
robotic fiber-positioning system based on AAO's Starbug technology,
(3) micro-assembly fore-optics required to optimally couple the Keck
II input beam to the fiber feed, (4) 10m-long fiber trains that
provide at least three focal-plane sampling options (single fibers,
multiple small IFUs, and a single large-format IFU), and (5) a bank
of three 4-channel spectrographs that provide a simultanous spectral
range of 0.31-1 $\mu$m at a resolution of $R\sim3500$ across the full
band.

\smallskip

% - Summarize the technology maturity of your implementation, listing the
%   demonstrated technologies and the technologies requiring development.

\noindent{\bf Technological maturity:} FOBOS is nearing the end of
its conceptual design phase, and will begin its preliminary design
phase in Oct 2020, contingent on funding. Components of this document
represent significant improvements in FOBOS's design since the
completion of our Astro2020 submission. Although requiring
significant design work, FOBOS's backend systems are built on
well-established technologies: FOBOS's spectrographs and fiber train
inherit significantly from SDSS, MaNGA, and DESI designs. FOBOS's
front-end leverages the more than 15 years of development of the
Starbugs position system, which have been deployed and are in the
final commissioning stages for the TAIPAN instrument on the AAT.
Further development of the Starbugs technology is needed for use in
FOBOS; however this represents a 2nd-generation implementation of the
technology and will leverage simultaneous development of the
GMT/MANIFEST instrument. Further development of the micro-assembly
fore-optics is also necessary; prototypes are currently being
developed with potential industry partners and tested at UCO/UCB.

\smallskip

% - Summarize areas where the data to support this RFI are not currently
%   available.

\noindent{\bf Limitations of this RFI:} Given this early stage in
development, particular project components with limited extant
data/information to address this RFI are as follows:

\newpage

\section{Science}

% - Briefly describe the scientific objectives and the most important
%   measurements required to fulfill these objectives. Feel free to refer
%   to science white papers or references from the literature.
% - Of the objectives, which are the most demanding? Why?
% - Present the highest-level technical requirements (e.g. spatial and
%   spectral resolution, sensitivity, timing accuracy) and their relation
%   to the science objectives.
% - For each performance requirement identified, describe as
%   quantitatively as possible the sensitivity of the science objectives
%   to achieve the requirement.  If you fail to meet a key requirement,
%   what would be the impact on achieving the science objectives?

\section{Enabling Technology}

% - Please provide information describing new enabling technologies
%   required for activity success.
% - Please indicate any non-US technology required for activity success
%   and what back up plans would be required if only US participation
%   occurred. 
% - For any technologies that have not been demonstrated by sub-scale or
%   full-scale models as of this request, please describe the rationale
%   for your technical maturity assessment, including the description of
%   analysis or hardware development activities to date, and its
%   associated technology maturation plan.
% - Describe the aspect of the enabling technology that is critical to the
%   concept’s success, and the sensitivity of mission performance if the
%   technology is not realized or is only partially realized.
% - Provide cost and schedule assumptions by year for all development
%   activities, and the efforts that allow the technology to be ready when
%   required, as well as an outline of readiness tests to confirm
%   technical readiness level.

% I don't think this is relevant to FOBOS:

\section{Telescope}

Not applicable.
 
% - Provide an overview description of the characteristics and
%   requirements of the optical telescope(s), antenna(s), or collector(s)
%   highlighting key capabilities and any residual technology risks. 
% - Provide diagrams or drawings showing the observatory or antenna array
%   with the instruments and other components labeled and a descriptive
%   caption.
% - Please provide any available review packages (e.g. Conceptual Design
%   Review, Preliminary Design Review) that describe the scope of
%   technical design and implementation.
% - Please describe any hardware/software with significant heritage.
% - Please fill out the table below regarding the primary scientific
%   equipment (e.g., Telescope or Antenna Array). Expand, contract, or
%   modify this table as necessary and applicable.
% - Identify and describe the three components of lowest technical
%   maturity, and explain how and when these components will be
%   demonstrated in hardware.
% - What are the three greatest risks to cost, schedule, and performance?
% - Describe any aspect of the design or implementation that may require
%   non-US participation.

\section{Instrumentation}

% - Describe the proposed science instrumentation, and briefly state the
%   rationale for selection. Discuss the capabilities of each instrument
%   (Inst #1, Inst #2 etc.) and how the instruments are used together.
%   Indicate whether cryogens or other cooling are required. 
% - Briefly describe any concept, feasibility, or definition studies
%   already performed and please provide any available Review Packages
%   (e.g., Conceptual Design Review, Preliminary Design Review) that
%   describe the instrument and its design and implementation.
% - Indicate the technical maturity level of the major elements and the
%   specific instrument maturity of the proposed instrumentation (for each
%   specific Inst #1, Inst#2 etc.), along with the rationale for the
%   assessment (i.e. examples of heritage, existence of breadboards,
%   prototypes, mass/volume and power comparisons to existing units, etc.
%   and any identifications of major long lead items). 
% - For instrument operations, provide a brief functional description of
%   operational modes, and calibration schemes. This can be documented in
%   the Operations Section. Describe the level of complexity associated
%   with analyzing the data to achieve the scientific objectives of the
%   investigation. Describe the types of data (e.g. bits, images).
% - Please fill out the table below regarding each instrument (if
%   applicable). Copy as needed for all instruments. Expand this table as
%   necessary and applicable.
% - What are the three primary technical issues or risks?
% - Describe the heritage of the instruments and associated sub-systems.
%   Indicate items that are to be developed, as well as any existing
%   hardware or design heritage.
% - Describe any instrumentation that may require non-US participation. 
% - List instruments to be delivered as part of construction versus
%   ongoing development when in operations.
% - Discuss anticipated data rates and volumes, as well as plans for
%   processing and archiving data.


\begin{center}
\footnotesize
\begin{tabular}{| l | r | r |}
\multicolumn{3}{c}{\large \bf FOBOS Instrument Table} \\
\hline
{\bf Item} & {\bf Value} & {\bf Units} \\
\hline
\hline
Type of instrument      & Multi-object spectrograph & \\\hline
Number of spectrographs & 3 & \\\hline
Multiplex & 1800 & single fibers \\
                        & 45 & 37-fiber IFU \\
                        & 1 & 1641-fiber IFU \\\hline
Field of view: Patrol   & 315 & arcmin$^2$ \\\hline
Field of view: Fiber    & 0.5 & arcsec$^2$ \\\hline
Field of view: Mini-IFU & 25 & arcsec$^2$ \\\hline
Field of view: Large-IFU & 705 & arcsec$^2$ \\\hline
Spectral range          & 0.31-1 & $\mu$m \\\hline
Spectral resolution     & 3500 & $R=\lambda/\delta\lambda$ \\\hline
Number of detectors     & 12 & 4 per spectrograph \\\hline
Detector size           & 6k $\times$ 6k & \\\hline
Thermal requirements    & & \\\hline
Size: Spectograph bank  & & m $\times$ m $\times$ m \\\hline
Size: Focal-plane module & & m $\times$ m $\times$ m \\\hline
Data volume             & $<$100 & Gb \\\hline
Development Schedule    & 100 & months \\\hline
\end{tabular}
\label{tab:instrument}
\end{center}

\section{Facilities}

% - If a site is not yet selected for the project, describe the
%   anticipated approach to conducting site studies, obtaining site
%   permissions, and executing environmental impact studies.
% - Please provide any available site plans. If such plans already address
%   items 2-5 below, these items do not need to be addressed separately.
% - Describe the site and its location, including size, altitude, access,
%   number of buildings, size of building(s) footprint and volume,
%   existing infrastructure, power, internet, environmental considerations
%   and logistics (proximity to major airport, housing and support for
%   construction crews and facility staff etc.).
% - Identify which facilities will be new and which facilities may be
%   pre-existing. Describe any existing facilities and their estimated
%   remaining useful life. Describe any upgrades to existing facilities
%   that will be undertaken. Describe any anticipated shared use of site
%   facilities between the concept being proposed and existing telescopes.
% - For antenna arrays, provide specific infrastructure required such as
%   concrete pad size, communications buildings, etc. for each element in
%   the array. For telescope mirrors, describe infrastructure needed for
%   mirror maintenance, e.g. coating facilities.
% - Describe atmospheric and radio frequency interference (RFI)
%   characteristics of the site insofar as they would affect observations
%   with the concept being proposed.

\section{Operations \& Observation Strategy}

% - Please provide operations plans or documents (Concept of Operations). 
% - Provide a description of the facility operations. For example: number
%   of staff required, position types, and 24-hour operation requirements.
% - Provide a description of the science operations. For example:
%   pre-observational planning, data processing/reduction required,
%   coordination with other facilities.
% - Discuss observatory efficiency, e.g. the impact of maintenance and
%   engineering and calibration time on faction of science time
%   availability.
% - Describe scope of engineering activities and time (day or night)
%   needed to maintain calibration and health of telescope/array and
%   instruments.
% - Summarize key software development and any science development
%   required.
% - Summarize any archiving requirements. 
% - Describe any high-level safety policies that will be required for
%   items of particular safety concern.

\section{Programmatic Issues \& Schedule}

% - Please provide a programmatic overview that describes the structure of
%   the overall organization including any international partners or
%   university partners etc. and any money or hardware they are providing.
%   Clearly indicate schedule and costs to date highlighting what has
%   already been delivered and/or clearly capturing progress to date, if
%   applicable.
% - Please describe any funding availability challenges and the total
%   impact it had, or may have, to cost and schedule.  Has all required
%   funding been committed?  If not, please explain.  Highlight in-kind
%   contributions in the past and planned from partners.  Clearly describe
%   what the NSF/DOE/Fed Gov funding request is and what it buys.
% - Please describe any unexpected challenges, key risks, and/or current
%   outstanding risks that require(d) significant mitigation which
%   affect/affected cost & schedule.
% - Describe the current top 3 risks to the development of the facility,
%   and proposed mitigation strategies. Please provide a top-level
%   schedule.

\section{Cost}

% - Provide a high-level facility unique Work Breakdown Structure (WBS)
%   with definitions. 
% - Fill out the cost estimate tables below using the broad bins provided.
%   Note the tables are divided into US Only and with International
%   Partners.  
% - Provide a basis of estimate for project WBS elements at the highest
%   levels (i.e., systems and major subsystems, as indicated in the cost
%   tables below), including internal overhead rates that are applied to
%   costs for labor.  Describe approach or methodology for highest cost
%   element estimates such as analogies, models, expert judgement,
%   construction bids, etc.  Describe cost assumptions made for each
%   element such as any production learning curves, labor and major
%   procurements.  Identify major subcontracted procurements and
%   associated vendors if available.  The “Prior” column is meant to be
%   actual costs incurred. 


%\newpage
%
%\setcounter{page}{1}
%\bibliographystyle{nsf.bst}
%\bibliography{references}

\end{document}


