
\documentclass[oneside,11pt]{amsart}

\usepackage{graphicx}
\usepackage{natbib,latexsym,url,enumitem,pdfpages}
\usepackage{color}
\usepackage{wrapfig}
\usepackage{caption}

% Some fancy commenting
\definecolor{todo}{RGB}{200,0,0}
\newcommand{\comment}[2][todo]{{\color{#1}[[{\bf #2}]]}}

\pretolerance=10000
\textwidth=6.4in
\textheight=8.95in
\voffset = 0.in
\topmargin=0.0in
\headheight=0.00in
\hoffset = 0.0in
\headsep=0.00in
\oddsidemargin=0in
\evensidemargin=0in
\parindent=2em
\parskip=0.2ex
 
\renewcommand{\baselinestretch}{1.03}

\special{papersize=8.5in,11in}

\newcommand{\markus}{\textcolor{green}}

\setlength{\parskip}{0.6 ex plus 0.4ex minus 0.2ex} \flushbottom
\pagestyle{plain} 

\begin{document}

\pagenumbering{arabic}

\begin{center}
\noindent {\sc Project Team}
\end{center}

\smallskip

\noindent We have built a strong team of astronomers, statisticians, and
engineers critical to the success of the Keck-FOBOS project, as follows:
%
\begin{itemize}
%
\item {\bf Kevin Bundy (UCO)} is the Principle Investigator, responsible
for the overall leadership and success of the project.\\[-5pt]
%
\item {\bf Kyle Westfall (UCO)} is the Project Scientist, responsible
for continued development and refinement of FOBOS's science
requirements, including coordinating science-team efforts.  He also
leads the development of the FOBOS data-management systems.\\[-5pt]
%
\item {\bf Nicholas MacDonald (UCO)} is the Project Manager, responsible
for maintaining the FOBOS development schedule and budget. He
coordinates development of the full project work breakdown structure and
project documentation.\\[-5pt]
%
\item {\bf Claire Poppett (SSL)} is the Instrument Scientist for the
front-end systems of the instrument. She is responsible for ensuring
that the atmospheric dispersion corrector, robotic positioning system,
and the fiber+microlens system meet the FOBOS science
requirements.\\[-5pt]
%
\item {\bf Renate Kupke (UCO)} is the Instrument Scientist for the
back-end systems of the instrument, responsible for ensuring all
multi-channel spectographs meet the FOBOS science requirements.\\[-5pt]
%
\item {\bf John O'Meara (WMKO)} is the Chief Scientist for the
W.~M.~Keck Observatory, and the PI of the WMKO sub-award. In
coordination with the Project Scientist, he ensures that the science
requirements of FOBOS also meet the larger scientific goals of
WMKO.\\[-5pt]
%
\item {\bf Marc Kassis (WMKO)} is the Instrument Program Manager at the
W.~M.~Keck Observatory, coordinating all its instrument development. He
coordinates with the FOBOS instrument team and the WMKO staff regarding
the FOBOS design and telescope-integration plan.\\[-5pt]
%
\item {\bf Lisa Hunter (UCSC)} is the Director of Institute for
Scientist and Engineer Educators (ISEE) and Director of the Akamai
Workforce Initiative (AWI) in Hawaii. She works with the instrument and
science teams to organize our Akamai internships and Professional
Development Programs.\\[-5pt]
%
\item {\bf Jon Lawrence (AAO-MQ)} is Head of Instrumentation of
Australian Astronomical Optics at Macquarie University and PI of the
AAO-Macquarie sub-award.  He and his team lead the design of the FOBOS
robotic positioning system based on their Starbugs technology.\\[-5pt]
%
\item {\bf Adam Bolton (NSF OIR Lab)} is Associate Director for the
Community Science and Data Center at NSF's OIR Lab.  As part of FOBOS's
community investment strategies, we will coordinate FOBOS efforts with
the broader efforts of the CSDC toward building tools that allow the
broader US community to capitalize on FOBOS public data, as well as
tools for planning observational programs to be executed via FOBOS's
open-access model.\\[-5pt]
%
\item {\bf Renbin Yan (UKy)} is an associate professor at the University
of Kentucky and the PI of its sub-award, who leads the design the FOBOS
calibration system.\\[-5pt]
%
\item {\bf Rachel Mandelbaum (CMU)} is a professor at Carnegie Mellon
University and the PI of its sub-award. She and fellow CMU Professor
{\bf Chad Schafer} lead the development of the data system used to plan
and optimize FOBOS observational programs.\\[-5pt]
%
\item {\bf Chad Schafer (CMU)} is the Data-Science Coordinator,
responsible for consulting on state-of-the-art solutions to FOBOS’s
data-science challenges and the efforts of the FOBOS science teams in
this regard. Schafer brings invaluable expertise as the co-chair of the
LSST Informatics and Statistics Science Collaboration.\\[-5pt]
%
\item {\bf Benjamin Williams (UW)} is a research professor at the
University of Washington and the PI of its sub-award. He coordinates the
design of the instrument-monitoring data system, as well as providing
significant contributions to the instrument simulator and data-reduction
software.
%
\end{itemize}

\noindent In addition to the instrument-development team, we have
convened three {\bf science teams}, one for each of the main science
themes of our proposal:
%
\begin{itemize}
%
\item {\bf Dark-Energy Science Team}: Led by Dan Masters (JPL).\\[-5pt]
%
\item {\bf Proto-Galaxy Ecosystem Science Team}: Led by Joe Burchett
(UCSC).\\[-5pt]
%
\item {\bf Local Group Archaeology Science Team}: Co-led by R. Michael
Rich (UCLA) and Benjamin Williams (UW).
%
\end{itemize}


% Lead Engineer, responsible for leading the optomechanical design of the instrument and working with WMKO staff to ensure successful deployment of FOBOS on the Keck II telescope.\\[-5pt]

%\newpage

\end{document}

