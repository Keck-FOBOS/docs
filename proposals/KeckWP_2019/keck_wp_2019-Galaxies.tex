%%%%
% -- Galaxies Science Cases
% --     FOBOS Keck White Paper 2019
%%%%

\subsection{The Galaxy Populations of Galaxy Clusters, and the
Globular Cluster Populations of Galaxies}

Presently, it is very expensive to conduct systematic spectroscopic
studies of the various galaxy types in rich galaxy clusters, like
Coma, due to their angular spread on the sky. With the flexible fiber
positioning afforded FOBOS and its large FOV, it will be possible to
simultaneously (and efficiently) build up an unprecedented library of
spectroscopic redshifts and stellar-population parameters of galaxies
in clusters as a function of redshift. Future development of a
multi-IFU front-end for FOBOS will allow us to obtain resolved
spectroscopy for 10s of these cluster galaxies, enabling us to
associate internal structures/properties of the galaxies with their
host cluster.

Likewise, probing the outskirts of nearby galaxies using globular
clusters (GCs) and/or planetary nebulae (PNe) is tedious and
inefficient with current facilities due to the sparse azimuthal
distribution of these chemodynamical tracers at large radii. FOBOS
will be a game changer in the study of the dynamical masses of
$\mathcal{M/M_\odot} \lesssim 10^{11}$ nearby galaxies, which
typically host $\lesssim1500$ GCs (Harris et al., 2013). With FOBOS,
it will be possible to simultaneously acquire spectra for a nearly
all GCs located to $\sim$50 kpc from their host galaxy in a single
night. These data will allow us to map the chemodynamics of massive
galaxy halos and infer orbital families as a function of
stellar-population properties. Additionally, we can more quickly
build statistically relevant models for GC formation in the context
of the larger galaxy population.

\subsection{Echos of Monsters}

Light echos from quasars.

\subsection{The volume density and chemistry of gas in between galaxies}

Not sure if there's anything new here.

\noindent\comment{X, Joe B.: Halos?  Anything new here}

\noindent\comment{KG, Joe H. - IGM tomography}

\subsection{The Proto-galaxy Ecosystem at $z$$\sim$2}
\label{sec:galaxies}

% From George:
% - fill out case for probing both galaxies and their “gas-filled
%   environments”
%    - make it more explicit that getting large numbers of redshifts
%      would make it possible to trace out large-scale structure in
%      detail
%    - enables studies of galaxy properties as a function of environment
%
% - also mention targeting galaxies along QSO lines of sight
%    - much higher target density than with LRIS, DEIMOS over larger FOV.
%
% - Worth discussing Lyman-alpha or metal-line tomography?  
%
% - More quantitative comparisons with existing data sets?
%    - What key science questions can FOBOS address that many years of
%      LRIS and DEIMOS observations have not been able to?  Surely some
%      level of the spectral tagging and photo-z training can be done
%      (and surely is being done) with existing data.  Is FOBOS going to
%      be a huge leap, or will it mainly be cleaning up neglected corners
%      of parameter space?
%
% - More excited to hear about how the FOBOS spectra will be used for
%   science directly, instead of support for LSST

%-----------------------------------------------------------------------

Understand the $z \sim 2$ galaxy ``ecosystem,'' including not only
the galaxies themselves but their gas-filled environments. The goal
is to build a comprehensive picture of the physical processes that
fuel proto-galaxy growth, shape their internal structure, and
influence their environment.

\noindent\comment{Cooper?} Build SDSS-like statistics for galaxies at
this key cosmic epoch. Exploit short spectroscopic exposures in
combination with photometry to provide environmental diagnostics for
1M galaxies at $z$=1--2. Photometric redshifts, while acceptable in
large cosmological analyses, wash out information about the local
position of galaxies with respect to one another. To characterize a
galaxy's local environment and identify its neighbors requires
(observationally expensive) spectroscopic redshifts. However, with
improved photo-$z$s available from Challenge \ref{photozs} and strong
priors on spectral types (Challenge \ref{phot}), the challenge here
is to push machine-learning techniques to deliver
\emph{spectroscopic} redshifts (with 300 km s$^{-1}$ accuracy) at the
lowest signal-to-noise possible. Reductions by factors of 4--5 in
exposure time would enable FOBOS to complete a 1M galaxy environment
survey at $z=1$--$2$ in just 20-30 nights.

\noindent\comment{Westfall, Bundy, Max -- Resolved spectroscopy}

\noindent\comment{Siana -- Lensed galaxies behing clusters}

\noindent\comment{Shapley, Siana: Additional section on Lyman-alpha continuum a $z$$\sim$3?}

